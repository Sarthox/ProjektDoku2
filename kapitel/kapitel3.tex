\newpage
\section{Analysephase}
\label{analysephase}

\subsection{Ist-Analyse}
\label{ist}
\begin{comment}
Wie eingangs in Kapitel \ref{projektumfeld} kurz erwähnt, verwendet \ac{SAZ} ein \ac{ERP}-System, welches regelmäßige Updates erhält und individuell angepasst werden kann. Durch die individuellen Anpassungen muss nach jedem Update, vor Produktivsetzung, ein Test durch die Anwender erfolgen. Die Resonanz dieser Testläufe, sowie die nachgelagerte Kommunikation aus Emails und ggf. Telefonaten, werden zurzeit über ein Exchange-Postfach verwaltet. Die eingehenden Rückmeldungen werden von den Entwicklern zu Aufgaben formuliert, anschließend kategorisiert, kommentiert und in einer Excel-Datei gelistet, welche als Grundlage zur weiteren Bearbeitung und Lösung der Problemstellungen dient.  Die Kategorisierung erfolgt auf Basis der Dringlichkeit. Tasks, die in der Produktivsetzung die Arbeit einschränken würden, werden vorrangig bearbeitet. Die Excel-Datei erhält zusätzlich eine farbliche Kategorisierung, sodass die Entwickler in kurzer Zeit einen Überblick erhalten, welche Tasks abgeschlossen, behoben, in Bearbeitung sind oder mit dem \ac{ERP}-Hersteller besprochen werden müssen. Die Liste beinhaltet noch weitere Informationen. Entwickler können für die einzelnen Tasks Kommentare, Workarounds und Testbeschreibungen verfassen und zusätzlich Dateipfade für weitere Materialien, wie z.B. Screenshots von Fehlermeldungen, hinterlegen. Einen hohen Stellenwert nimmt dabei die Spalte der Kommentare ein, denn häufig erfolgen Rückfragen durch die Entwickler, da die Rückmeldungen der Anwender keiner einheitlichen Form folgen und teilweise notwendige Informationen zur Problemlösung fehlen. Zu beachten ist, dass auch zur aktuellen produktiv laufenden Version, Anfragen von Anwendern eintreffen, die nach dem identischen Schema verarbeitet werden. Diese Vorgehensweise, der Protokollierung in einer Excel-Datei, Kommunikation über ein Exchange-Postfach und die nicht einheitliche Form der Meldung bindet unnötig Ressourcen und verlängert die Projektzeit teilweise erheblich.	
\end{comment}


\subsection{Wirtschaftlichkeitsanalyse}
\label{wirtschaftlichkeitsanalyse}

\subsection{"Make or Buy"- Entscheidung}
\label{makeOrBuy}

\subsection{Projektkosten}
\label{projektkosten}

\subsection{Amortisationsdauer}
\label{amortisationsdauer}

\subsection{Anwendungsfälle}
\label{anwednungsfaelle}

\subsection{Anforderungen}
\label{anforderungen}


