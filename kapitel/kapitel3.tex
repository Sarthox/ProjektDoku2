\newpage
\section{Analysephase}
\label{analysephase}

\subsection{Ist-Analyse}
\label{ist}
\begin{comment}
evtl alte otrs mit nutzwertanalyse vergleichen?
\end{comment}
Die \ac{KM} bietet \ac{OTR} für Makler an. Zur Zeit gibt es mehrere Generationen an Rechnern. Die älteste ist mit PHP umgesetzt und hat eine altmodische Benutzeroberfläche. Eine neuere Tarifrechnerbasis ist ähnlich wie diese Neuentwicklung aufgebaut auf einem .NET-Backend und einem Vue.js Frontend. Allerdings stellte sich bei dem Einführen neuer Tariflinien heraus, dass diese sich nicht einfach in diesen \ac{OTR} einbinden ließ. \\
Demnach existieren funktionale \ac{OTR}, die beim Vertrieb von Versicherungsprodukten helfen, allerdings sind diese entweder für den Endnutzer nicht anschaulich oder modern und/oder für den Entwickler schwierig zu warten oder zu erweitern.

\subsection{Wirtschaftlichkeitsanalyse}
\label{wirtschaftlichkeitsanalyse}

\subsection{"Make or Buy"- Entscheidung}
\label{makeOrBuy}

\subsection{Projektkosten}
\label{projektkosten}

\subsection{Amortisationsdauer}
\label{amortisationsdauer}

\subsection{Anwendungsfälle}
\label{anwednungsfaelle}

\subsection{Anforderungen}
\label{anforderungen}


