\newpage
\section{Analysephase}
\label{analysephase}

\subsection{Ist-Analyse}
\label{ist}
\begin{comment}
evtl alte otrs mit nutzwertanalyse vergleichen?
\end{comment}
Die \ac{KM} bietet \ac{OTR} für Makler an. Zur Zeit gibt es mehrere Generationen an Rechnern. Die älteste ist mit PHP umgesetzt und hat eine altmodische Benutzeroberfläche. Eine neuere Tarifrechnerbasis ist ähnlich wie diese Neuentwicklung aufgebaut auf einem .NET-Backend und einem Vue.js Frontend. Allerdings stellte sich bei dem Einführen neuer Tariflinien heraus, dass diese sich nicht einfach in diesen \ac{OTR} einbinden ließ. \\
Demnach existieren funktionale \ac{OTR}, die beim Vertrieb von Versicherungsprodukten helfen, allerdings sind diese entweder für den Endnutzer nicht anschaulich oder modern und/oder für den Entwickler schwierig zu warten oder zu erweitern.

\subsection{Wirtschaftlichkeitsanalyse}
\label{wirtschaftlichkeitsanalyse}

\subsection{"Make or Buy" - Entscheidung}
\label{makeOrBuy}
Dadurch, dass die \ac{KM} neue und innovative Versicherungskonzepte auf den Markt bringt, eignen sich sogenannte Baukastenrechner nicht für diese Tarife. Die Vorlagen dieser Rechner sind nicht so individuell gestaltbar wie die \ac{KM} es benötigt.\\
Da es schon ältere Generationen an Tarifrechner gibt, wäre auch eine Überlegung einfach eine ältere Tarifrechnerbasis weiter zu benutzen, anstatt einen neuen einzuführen. Die älteste Generation ist allerdings technologisch nicht tragbar und muss grundlegend erneuert werden. Ein neuer \ac{OTR} sollte die neue Grundlage werden, allerdings war dieser Versuch nicht erfolgreich.\\
Der neue \ac{OTR} soll ein universeller Ansatz werden um alle zukünftigen Produkte gleich abzubilden.

\subsection{Projektkosten}
\label{projektkosten}
Die Kosten des Gesamtprojekts, also der kompletten \ac{OTR} Basis, belaufen sich auf ca. 80.000-100.000€. Dazu gehören allerdings mehrere Nebeneffekte die aus dem Gesamtprojekt  resultieren. So wurde im Rahmen des \ac{OTR}s die \ac{cicd} Pipeline von Jenkins auf \gls{gitlab} umgestellt. Außerdem wurde ein einheitliches Corporate-Design eingeführt, an welchem die Marketingabteilung gearbeitet hat. Der erste Tarif in dem \ac{OTR} wird eine Betriebshaftpflicht geplant, welche ausgearbeitet werden musste und anhand welcher alle tarifspezifischen Komponenten des Rechners implementiert wurden. Dazu kommt, dass das Bestandsführungssystem für die Betriebshaftpflicht konfiguriert werden muss.\\

Die Kosten der Fußzeilenkomponente lässt sich anhand der aufgewendeten Stunden berechnen.
Dabei wird eine Arbeitsstunde eines Auszubildenden mit 100€ berechnet. Weitere benötigte Mitarbeiter, z.B. für Absprachen, lassen sich mit durchschnittlich 130€ pro Stunde berechnen. 
Demnach ergibt sich daraus:
(69 Std * 100€/Std)Azubi = 6900€
 + 
 (1 Std * 4 * 130€/Std)Dailys = 520€
 + 
 2 Std * 6 * 130€/Std Weeklys = 1560€
 5 Std * 2 *  130€/Std AG gespräche = 1300€
 = 10.280€
 In Meetings auch andere Komponenten besrpochen??? Vllt weniger meetings eintragen deswegen??
\subsection{Amortisationsdauer}
\label{amortisationsdauer}

\subsection{Anwendungsfälle}
\label{anwednungsfaelle}

\subsection{Anforderungen}
\label{anforderungen}


