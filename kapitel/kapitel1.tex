\newpage
% Einleitung

\section{Einleitung}
\label{einleitung}

\subsection{Projektumfeld}
\label{projektumfeld}
\textbf{Mariosoft}
\\

Bei der Mariosoft handelt es sich um eine IT-Abteilung der \ac{KM}. Die Mariosoft besteht derzeit aus 18 Mitarbeitern und beschäftigt sich mit der Konzeption, Programmierung und Wartung von internen Softwarelösungen wie Online Tarifrechnern, Kundenverwaltung und Übersichtsseiten für Makler.
\\
Das umzusetzende Projekt ist im Umfeld der Mariosoft angesiedelt und ist eine Komponente für eine gleichzeitig entstehende Webanwendung.

\textbf{\acl{KM}}
\\

Bei der \ac{KM} handelt es sich um einen Finanzdienstleister der Versicherungskonzepte entwickelt und diese auch Komplett von der Annahme bis hin zur Schadensregulierung verwaltet.\\
Das Unternehmen, bestehend derzeit aus 119 Mitarbeitern, bietet Dienstleistungen sowie Portale für Versicherer, Makler und Kunden an.

\subsection{Projektziel}
\label{projektziel}
Ziel des Gesamtprojekts ist ein neuer, modularer \ac{OTR} als Webseite, bzw. \ac{SPA} . Dieser soll so entworfen werden, dass die einzelnen Komponenten austauschbar sind und für weitere Versicherungsprodukte benutzt werden können. Außerdem soll die Benutzeroberfläche einheitlich gestaltet werden. Allgemein soll die Einführung neuer, und die Erweiterung bestehender Versicherungsprodukte vereinfacht werden. Der \ac{OTR} wird bestehen aus einem .NET 6.0 Backend, welches viele Schnittstellen zu umliegenden, firmeninternen Systemen, sowie eine Cachespeicherung besitzt und einem Vue.js 3.0 Frontend mit Designvorlagen von Vuetify, sowie einer eigenen Oberflächenkomponentenbibliothek welche Vuetifyelemente kapselt und im Corporatedesign zur Verfügung stellt. Im Frontend benutzen wir Typescript als Programmiersprache.\\
Die Kommunikation zwischen Front-, und Backend wird durch REST-Apiaufrufen realisiert. Um die Daten synchron zu halten wird jede Änderung im Frontend in den Piniastore von Vue.js, ein Speicher welcher den Localstorage des Browsers benutzt, geschrieben und zeitgleich per JSON-Patch API-Aufruf ans Backend weitergesendet, welches die neuen, oder geänderten Daten im Cache speichert und wenn möglich eine Berechnung, sowie Validierung durchführt und die berechneten Beiträge und Validierungsergebnisse zurück sendet. Alle Schnittstellen sowie \ac{DTO} werden beim Bauen des Backendprojektes automatisch für das Frontend generiert.\\
Als Versionsverwaltung für dieses Projekt wird Git, mit GitFlow Workflow, genutzt, speziell der Service GitLab. Mit GitLab wird außerdem eine \ac{CICD} Pipeline realisiert, also eine Möglichkeit automatisch nach jeder Änderung das Projekt zu bauen, zu testen und zu veröffentlichen. Veröffentlicht wird der \ac{OTR} in einem Kubernetes-Cluster, also wird das Projekt vorher als Dockerimage gebaut um es dann in einem Container laufen zu lassen. Für jeden Feature-, Develop- und Masterbranch im Git wird jeweils eine eigene Instanz im Kubernetes Cluster erstellt, so dass man jeden Stand auf dem Testsystem testen kann.
\\\\
Bei einem \ac{OTR} hat ein Makler die Möglichkeit Angaben zu einem Versicherungstarif zu treffen und sich anhand dessen Beiträge ausrechnenzulassen. Außerdem sind hilfreiche Funktionalitäten gegeben, wie z.B. das Ausdrucken von Anträgen und Angeboten, das Speichern von Angeboten, um diese später wieder laden und weiter bearbeiten zu können und das Versenden von tarifrelevanten Dokumenten per E-Mail. Diese Funktionalitäten werden in der Fußzeile in Form von Knöpfen angeboten, um dessen Implementierung dieses Projekt handelt. Die Fußzeile beinhaltet außerdem Angaben zum Copyright und einen Verweis zum Impressum.
\\\\

%Angebotsspeicherung
Beim Speichern eines Angebotes sollen alle im \ac{OTR} eingegebenen Daten, so wie, wenn anhand der Angaben möglich, berechnete Beiträge im sogennanten Maklerportal für den angemeldeten Makler gespeichert werden. Das Maklerportal ist eine Webseite an der sich Makler anmelden können wo sie Zugriff auf Tarifrechner haben, Übersichten über ihr Kundenbestand und noch vieles mehr. Zu dem können dort also auch gespeicherte Angebote dort eingesehen, bearbeitet und mit einem \ac{OTR} aufgerufen werden. Beim Speichern wird über mehrere anliegende Systeme ein Angebot in einer Datenbank gespeichert. Wichtig ist dabei die Session, welche von einem der Systeme generiert wird, realisiert als \ac{GUID}. Mithilfe dieser wird ein Angebot im XML-Format in einer Datenbank gespeichert und ist dadurch eindeutig identifizierbar.\\
Außerdem ist es vorgesehen, dass Makler Angebote kopieren können. Dies beinhaltet die gleiche Logik wie die Angebotsspeicherung, bis auf dass eine neue \ac{GUID} erzeugt wird, wodurch dementsprechend ein neuer Eintrag in der Datenbank und im Maklerportal hinterlegt wird. Außerdem muss mit der \ac{GUID}, bzw. Session der Kopie weitergearbeitet werden.\\
Die \ac{KM} bietet auch \ac{OTR} außerhalb des Maklerportals an. Für diese Rechner ist dann keine Angebots-, oder Kopiespeicherung vorgesehen, dementsprechend sind die Knöpfe dort nicht vorhanden. \\

% Drucken
Die Druckenfunktionalität wird durch zwei Knöpfe abgebildet, einen um ein Angebot zu drucken, der zweite für den Antrag. Ein Angebot wird z.B. dem Endkunden vom Makler unterbreitet und ein Antrag ist ein verbindliches Dokument mit welchem ein Versicherungsvertrag abgeschlossen werden kann. Somit ist diese Funktionalität sehr wichtig, als auch die Korrektheit der Daten in diesen Dokumenten. Ein Dokument soll dann gedrückt werden können, wenn Beiträge erfolgreich berechnet wurden. Dafür müssen alle berechnungsrelevanten Angaben getroffen worden sein. Die Knöpfe sind bis das geschehen ist deaktiviert und werden reaktiv aktiviert. Außerdem kann es sein, dass für einen Tarif der Angebot- oder Antragsdruck deaktiviert ist, wo dann natürlich die Knöpfe auch dementsprechend aktiv sind. Das Drucken ist dabei eigentlich das Herunterladen eines PDF-Dokumentes, damit der Makler dieses auch per E-Mail verschicken oder selber ausdrucken kann. Das Dokument wird von einem angebundenem System anhand der Daten in XML-Format generiert.

%Versand
Eine weitere Funktion ist die des Dokumentenversands. Bei Drücken dieses Knopfs öffnet sich ein Modal wo der Nutzer die gewünschten Dokumente aus allen relevanten Dokumenten auswählen kann. Dazu muss noch eine, oder mehrere E-Mails angegeben werden und dann werden die Dokumente per E-Mail in einem ZIP-Archiv versendet. Die Dokumente beinhalten rechtliche Informationen und der Antrag, sowie das Angebot können auch ausgewählt werden, mit der gleichen Einschränkung wie bei der Druckfunktionalität beschrieben. Die komplette Dokumentenversandkomponente gibt es als eigenständige Bibliothek, da sie auch für andere \ac{OTR} genutzt wird, so dass diese für Frontend nur eingebunden werden musste.

\subsection{Projektbegründung}
\label{projektbegründung}
Die Motivation hinter diesem Projekt begründet sich in der einfachen Erweiterbarkeit für neue Versicherungsprodukte, so wie eine Überholung des Aussehens des \ac{OTR}. Allgemein ist der \ac{OTR} durch den modularen und komponentenbasierten Aufbau des Projekts besser wartbar und Einzelteile einfacher auszutauschen. \\
Die Fußzeile ist eine Komponente die in allen bestehenden Produkten und zukünftig kommenden Produkten vorhanden sein muss. Die Position bietet sich außerdem für die gegeben Funktionalitäten gut an, da sie unabhängig von dem Kontext der Rest der Webseite funktionieren soll.
\subsection{Projektschnittstellen}
\label{projektschnittstellen}
Das Backend hat %, wie schon erwähnt,
Schnittstellen zu einigen Systemen. Es gibt das \ac{TG}, welches den Einstiegspunkt von Extern zu den Tarifrechnern darstellt. Dieses, sowie der \ac{OTR} benutzt das \ac{TRG}, welches als REST-Schnittstelle funktioniert um unter anderem Konfigurationen für Tarife oder das Speichern von Angeboten anzubieten. Das \ac{TRG} speichert Daten in eine MS SQLServer Datenbank, während der neue \ac{OTR} keine direkte Anbindung an eine Datenbank hat. \\
Weitere Schnittstellen und Systeme die von der Fußzeilenkomponente genutzt werden ist der Mailservice, eine Anwendung welche E-Mails verschickt, sowie der PDFToolsservice, welcher die Generierung von PDF-Dokumenten übernimmt. Die Fußzeile, so wie alle Komponenten im Frontend bekommen ihre Daten über den Vue.js Pinia Store geliefert.
Der \ac{OTR} wird vor allem von Maklern benutzt um Tarife für ihre Endkunden auszurechnen.
\subsection{Projektabgrenzung}
\label{projektabgrenzung}
Das bearbeitete Projekt ist hierbei nur die Fußzeile mit den Knöpfen einschließlich ihrer Funktionalitäten. Eine Benutzeroberflächenkomponentenbibliothek ist gegeben, aus welcher die Knöpfe eingebunden werden. Genau so ist die vorher beschriebene Synchronisation durch JSONPatch schon vorhanden, so wie ein Backend mit mehreren Schnittstellen welche nur für die einzelnen Funktionalitäten erweitert werden müssen. Im Frontend ist ein Gridlayout mit leeren Elementen gegeben.