\newpage
% Einleitung

\section{Einleitung}
\label{einleitung}

\subsection{Projektumfeld}
\label{projektumfeld}
\textbf{\acl{KM}}
\\

Bei der \ac{KM} handelt es sich um einen Finanzdienstleister der Versicherungskonzepte entwickelt und diese auch Komplett von der Annahme bis hin zur Schadensregulierung verwaltet.\\
Das Unternehmen, bestehend derzeit aus 118 Mitarbeitern, bietet Dienstleistungen sowie Portale für Versicherer, Makler und Kunden an.

\textbf{Mariosoft}
\\

Bei der Mariosoft handelt es sich um eine IT-Abteilung der \ac{KM}. Die Mariosoft besteht derzeit aus 18 Mitarbeitern und beschäftigt sich mit der Konzeption, Programmierung und Wartung von internen Softwarelösungen wie \ac{OTR}, Kundenverwaltung und Übersichtsseiten für Makler.
\\
Das umzusetzende Projekt ist im Umfeld der Mariosoft angesiedelt und ist eine Komponente für eine gleichzeitig entstehende Webanwendung.

\subsection{Projektziel}
\label{projektziel}
Ziel des Gesamtprojekts ist ein neuer, modularer \ac{OTR} als Webseite bzw. \ac{SPA}. Dieser soll so entworfen werden, dass die einzelnen Komponenten austauschbar sind und für weitere Versicherungsprodukte benutzt werden kann. Außerdem soll die Benutzeroberfläche einheitlich gestaltet werden. Allgemein soll die Einführung neuer und die Erweiterung bestehender Versicherungsprodukte vereinfacht werden. Der \ac{OTR} wird bestehen aus einem .NET 6.0 Backend, welches unter anderem viele Schnittstellen zu umliegenden, firmeninternen Systemen, einen Cachespeicher und weitere Logik auf welche im Verlauf dieser Dokumentation eingegangen wird, besitzt. Das Frontend besteht aus \gls{vue} mit Designvorlagen von \gls{vuetify} sowie einer eigenen Oberflächenkomponentenbibliothek, welche \gls{vuetify}elemente kapselt und im Corporatedesign zur Verfügung stellt. Im Frontend benutzen wir \gls{typescript} als Programmiersprache.
\\
Bei einem \ac{OTR} hat ein Makler die Möglichkeit Angaben zu einem Versicherungstarif zu treffen und sich anhand dessen Beiträge ausrechnen zulassen. Außerdem sind hilfreiche Funktionalitäten gegeben, wie z.B. das Ausdrucken von Anträgen und Angeboten, das Speichern von Angeboten, um diese später wieder laden und weiter bearbeiten zu können und das Versenden von tarifrelevanten Dokumenten per E-Mail. Diese Funktionalitäten werden in der Fußzeile in Form von Schaltflächen angeboten. Die Implementierung dieser Fußzeile wird von mir im Rahmen dieses Projekts umgesetzt. Die Fußzeile beinhaltet außerdem Angaben zum Copyright und einen Verweis zum Impressum.
\\

%Angebotsspeicherung
Beim Speichern eines Angebots sollen alle im \ac{OTR} eingegebenen Daten im sogenannten Maklerportal für den angemeldeten Makler gespeichert werden. Diese können über das Maklerportal wieder geladen werden. Außerdem ist es vorgesehen, dass Makler Angebote kopieren können.
Die \ac{KM} bietet auch \ac{OTR} außerhalb des Maklerportals an. Für diese Rechner ist dann keine Angebots- oder Kopiespeicherung vorgesehen, dementsprechend sind die Schaltflächen dort nicht vorhanden. \\

% Drucken
Die Druckenfunktionalität wird durch zwei Schaltflächen abgebildet, einen um ein Angebot zu drucken, der zweite für den Antrag. Ein Angebot wird z.B. dem Endkunden vom Makler unterbreitet und ist, genau wie der Antrag, ein verbindliches Dokument mit welchem ein Versicherungsvertrag abgeschlossen werden kann. Die Funktionalität wird häufig genutzt und ist damit wichtig. Ein Dokument soll dann gedruckt werden können, wenn Beiträge erfolgreich berechnet wurden. Dafür müssen alle berechnungsrelevanten Angaben getroffen worden sein.
Die Schaltflächen sind bis das geschehen ist deaktiviert und werden reaktiv aktiviert. Außerdem kann es sein, dass für einen Tarif der Angebot- oder Antragsdruck deaktiviert ist. Dementsprechend sind die Schaltflächen aktiviert, oder nicht aktiviert.  Das Drucken ist dabei eigentlich das Herunterladen eines PDF-Dokumentes, damit der Makler dieses auch per E-Mail verschicken oder selber ausdrucken kann. Das Dokument wird von einem angebundenem System anhand der Daten in XML-Format generiert.

%Versand
Eine weitere Funktion ist die des Dokumentenversands. Bei Drücken dieser Schaltfläche öffnet sich ein Dialog, wo der Nutzer die gewünschten Dokumente aus allen relevanten Dokumenten auswählen kann um sie zu versenden. Dazu muss noch eine oder mehrere E-Mails angegeben werden und dann werden die Dokumente per E-Mail in einem ZIP-Archiv versendet. Die Dokumente beinhalten rechtliche Informationen und den Antrag sowie das Angebot. Die komplette Dokumentenversand-Komponente gibt es als eigenständige Bibliothek, da sie auch für andere \ac{OTR} genutzt wird, so dass diese für das Frontend eingebunden werden musste.

\subsection{Projektbegründung}
\label{projektbegründung}
Die Motivation hinter diesem Projekt begründet sich in der einfachen Erweiterbarkeit für neue Versicherungsprodukte sowie ein Redesign des Aussehens des \ac{OTR}. Der \ac{OTR} ist durch den modularen und komponentenbasierten Aufbau des Projekts besser wartbar. \\
Die Fußzeile ist eine Komponente die in allen bestehenden Produkten und zukünftig kommenden Produkten vorhanden sein muss. Die Platzierung der gegebenen Funktionalitäten in die Fußzeile bietet sich dafür gut an, da sie unabhängig von dem Kontext der Rest der Webseite funktionieren soll.
\subsection{Projektschnittstellen}
\label{projektschnittstellen}
Das Backend hat Schnittstellen zu einigen Systemen. Es gibt das \ac{TG}, welches den Einstiegspunkt von Extern zu den Online Tarifrechnern darstellt. Dieses sowie der \ac{OTR} benutzt das \ac{TRG}, welches als REST-Schnittstelle funktioniert um unter anderem Konfigurationen für Tarife oder das Speichern von Angeboten anzubieten.
Eine weitere Schnittstelle die von der Fußzeilenkomponente genutzt werden ist der PDFToolsservice, welcher die Generierung von PDF-Dokumenten übernimmt. Die Fußzeile sowie alle Komponenten im Frontend bekommen ihre Daten über den \gls{localstorage} des Browsers (Vue.js Pinia Store) geliefert.\\
Das Maklerportal ist eine Webseite an der sich Makler anmelden können, wo sie den \ac{OTR} aufrufen können, Übersichten über ihr Kundenbestand haben und noch vieles mehr. Zu dem können dort auch gespeicherte Angebote eingesehen, bearbeitet und mit einem \ac{OTR} aufgerufen werden.\\
Der \ac{OTR} wird vor allem von Maklern benutzt um Tarife für ihre Endkunden auszurechnen.


\subsection{Projektabgrenzung}
\label{projektabgrenzung}
Das bearbeitete Projekt ist hierbei nur die Fußzeile mit den Schaltflächen einschließlich ihrer Funktionalitäten. Eine Benutzeroberflächenkomponentenbibliothek ist gegeben, aus welcher die Schaltflächen eingebunden werden. Genauso ist die vorher beschriebene Synchronisation durch HTTP-JSON-Patch schon vorhanden. Weiterhin ist ein Backend mit mehreren Schnittstellen welche nur für die einzelnen Funktionalitäten erweitert werden müssen verfügbar. Im Frontend ist ein Gridlayout mit Elementvorlagen gegeben, welche genauso erweitert werden müssen.