% Einleitung

\section{Einleitung}
\label{einleitung}

\subsection{Projektumfeld}
\label{projektumfeld}
\textbf{\acl{KM}}
\\
Bei der \ac{KM} handelt es sich um einen Finanzdienstleister der Versicherungskonzepte entwickelt. Die Konzepte werden über Partnerschaften mit namhaften Versicherern auf den Markt gebracht. \ac{KM} übernimmt die vollständige Verwaltung, vom Vertrieb über freie Makler über die Vertragsannahme, bis zur Schadenregulierung.
Das Unternehmen, bestehend derzeit aus 118 Mitarbeitern, bietet Dienstleistungen sowie Portale für Versicherer, Makler und Kunden an.

\textbf{Mariosoft}
\\
Bei der Mariosoft handelt es sich um eine IT-Abteilung der \ac{KM}. Die Mariosoft besteht derzeit aus 18 Mitarbeitern und beschäftigt sich mit der Konzeption, Programmierung und Wartung von internen Softwarelösungen wie \ac{OTR}, Kundenverwaltung und Übersichtsseiten für Makler.
\\
Das umzusetzende Projekt ist im Umfeld der Mariosoft angesiedelt und ist eine Komponente für eine gleichzeitig entstehende Webanwendung.

\subsection{Projektziel}
\label{projektziel}
Ziel des Gesamtprojekts ist ein neuer, modularer \ac{OTR} als Webseite bzw. \ac{SPA}. Dieser soll so entworfen werden, dass die einzelnen Komponenten austauschbar sind und für weitere Versicherungsprodukte benutzt werden können. Außerdem soll die Benutzeroberfläche einheitlich gestaltet werden. Allgemein soll die Einführung neuer und die Erweiterung bestehender Versicherungsprodukte vereinfacht werden. Der \ac{OTR} wird bestehen aus einem \gls{net} 6.0 Backend, welches Schnittstellen zu umliegenden firmeninternen Systemen hat. Des Weiteren ist ein Cachespeicher und weitere Logik vorhanden. Das Frontend besteht aus \gls{vue} mit Designvorlagen von \gls{vuetify} sowie einer eigenen Oberflächenkomponentenbibliothek, welche \gls{vuetify}elemente kapselt und im \gls{corpdes} zur Verfügung stellt. Im Frontend wird \gls{typescript} als Programmiersprache benutzt.\\
Als Versionsverwaltung für dieses Projekt wird \gls{git} mit \gls{gitflow} genutzt. Das \ac{CICD} wird mittels \gls{gitlab} Pipelines realisiert, also eine Möglichkeit automatisch nach jeder Änderung das Projekt zu bauen, zu testen und zu veröffentlichen. Veröffentlicht wird der \ac{OTR} in einem \gls{k8s}-Cluster. Im Abschnitt \ref{zielplattform} \nameref{zielplattform} wird auf das Verfahren weiter eingegangen.
\\
Bei einem \ac{OTR} hat ein Makler die Möglichkeit Beiträge für versicherte Risiken zu errechnen. Darüber hinaus sind hilfreiche oder sogar verpflichtende Funktionalitäten gegeben, wie z.B. das Ausdrucken von Anträgen und Angeboten, das Speichern von Angeboten, um diese später wieder laden und weiter bearbeiten zu können und das Versenden von tarifrelevanten Dokumenten per E-Mail. Diese Funktionalitäten werden in der Fußzeile in Form von Schaltflächen angeboten. Die Implementierung dieser Fußzeile wird von mir im Rahmen dieses Projekts umgesetzt. Die Fußzeile beinhaltet außerdem Angaben zum Copyright und einen Verweis zum Impressum.
\\

%Angebotsspeicherung
Beim Speichern eines Angebots sollen alle im \ac{OTR} eingegebenen Daten im sogenannten Maklerportal für den angemeldeten Makler gespeichert werden. Diese können über das Maklerportal wieder geladen werden. Des Weiteren ist es vorgesehen, dass Makler Angebote kopieren können.
Die \ac{KM} bietet auch \ac{OTR} außerhalb des Maklerportals an. Für diese Rechner ist dann keine Angebots- oder Kopiespeicherung vorgesehen, dementsprechend sind die Schaltflächen dort nicht vorhanden. \\
% Drucken
Die Druckenfunktionalität wird durch zwei Schaltflächen abgebildet, eine um ein Angebot zu drucken, die zweite für den Antrag. Ein Angebot wird z.B. dem Kunden vom Makler unterbreitet und ist, genau wie der Antrag, ein verbindliches Dokument mit welchem ein Versicherungsvertrag abgeschlossen werden kann. Die Funktionalität wird häufig genutzt und ist damit wichtig. Die Erzeugung von Angebots- und Antragsdokumenten setzt die Berechnung der Risiken voraus. Dafür müssen alle berechnungsrelevanten Angaben getroffen worden sein. Nur bei einer gültigen Berechnung werden die Schaltflächen aktiv. Außerdem kann es sein, dass für einen Tarif der Angebots- oder Antragsdruck deaktiviert ist. Dementsprechend sind die Schaltflächen aktiviert oder deaktiviert. Das Drucken ist dabei eigentlich das Herunterladen eines PDF-Dokumentes. Das Dokument wird von einem angebundenem System anhand der Daten in \gls{xml}-Format generiert.\\
%Versand
Eine weitere Funktion ist die des Dokumentenversands. Beim Drücken dieser Schaltfläche öffnet sich ein Dialog, in dem der Nutzer die gewünschten Dokumente aus allen relevanten Dokumenten auswählen kann um sie zu versenden. Dazu gehören auch die zuvor beschriebenen Angebots- und Antragsdokumente. Dazu müssen noch eine oder mehrere E-Mailadressen angegeben werden und dann werden die Dokumente an diese E-Mailadressen in einem ZIP-Archiv versendet. Die Dokumente beinhalten rechtliche Informationen und den Antrag sowie das Angebot. Die komplette Dokumentenversand-Komponente gibt es als eigenständige Bibliothek, da sie auch für andere \ac{OTR} genutzt wird, so dass diese für das Frontend eingebunden werden musste.\\
Die beschriebenen Schaltflächen sind als Anwendungsfälle in einem Anwendungsfalldiagramm (s. A.\ref{usecasefooter} \nameref{usecasefooter}) dargestellt.

\subsection{Projektbegründung}
\label{projektbegründung}
Die Motivation hinter diesem Projekt begründet sich in der einfachen Erweiterbarkeit für neue Versicherungsprodukte sowie ein Redesign des Aussehens des \ac{OTR}. Der \ac{OTR} ist durch den modularen und komponentenbasierten Aufbau des Projekts besser wartbar. \\
Die Fußzeile ist eine Komponente die in allen bestehenden Produkten und zukünftig kommenden Produkten vorhanden sein muss. Die Platzierung der gegebenen Funktionalitäten in der Fußzeile bietet sich dafür gut an, da sie unabhängig von der übrigen Webseite funktionieren soll.
\subsection{Projektschnittstellen}
\label{projektschnittstellen}
Das Backend hat Schnittstellen zu einigen Systemen. Es gibt das \ac{TG}, welches den Einstiegspunkt von Extern zu den \ac{OTR} darstellt. Dieses sowie der \ac{OTR} benutzt das \ac{TRG}, welches als REST-Schnittstelle funktioniert um unter anderem Konfigurationen für Tarife oder das Speichern von Angeboten anzubieten.
Eine weitere Schnittstelle die von der Fußzeilenkomponente genutzt wird, ist der PDFToolsservice, welcher die Generierung von PDF-Dokumenten übernimmt. Die Fußzeile sowie alle Komponenten im Frontend bekommen ihre Daten über den \gls{localstorage} des Browsers (\gls{vue} \gls{pinia}) geliefert.\\
Das Maklerportal ist eine Webseite an der sich Makler anmelden und den \ac{OTR} aufrufen können, Übersichten über ihren Kundenbestand haben und noch vieles mehr. Außerdem steuert das Maklerportal die Konfiguration der \ac{OTR}. Zudem können dort auch gespeicherte Angebote eingesehen, bearbeitet und mit einem \ac{OTR} aufgerufen werden.\\
Der \ac{OTR} wird vor allem von Maklern benutzt um Tarife für ihre Kunden auszurechnen.


\subsection{Projektabgrenzung}
\label{projektabgrenzung}
Das bearbeitete Projekt ist hierbei die Fußzeile mit den Schaltflächen einschließlich ihrer Funktionalitäten. Eine Benutzeroberflächenkomponentenbibliothek ist gegeben, aus welcher die Schaltflächen eingebunden werden. Genauso ist eine Synchronisation zwischen Front- und Backend durch HTTP-JSON-PATCH schon vorhanden. Weiterhin ist ein Backend mit mehreren Schnittstellen, welche für die einzelnen Funktionalitäten erweitert werden müssen, verfügbar. Im Frontend ist ein Gridlayout mit Elementvorlagen gegeben, welche genauso erweitert werden müssen.