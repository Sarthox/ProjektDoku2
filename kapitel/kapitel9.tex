\section{Projektabschluss}
\label{Projektabschluss}

\subsection{Abnahme}
\label{Abnahme}
Durch die regelmäßigen Absprachen und Codereviews durch das Entwicklerteam sowie die wöchentlichen Absprachen bezüglich der Oberfläche gab es keine formale Abnahme der Fußzeilenkomponente. Alle anfallenden Anforderungen konnten im Laufe der Implementierung umgesetzt werden. Ausgenommen davon sind Anforderungen die kurz vor Abschluss des Projekts aufgekommen sind. Auf diese wird in \ref{ausblick} weiter eingegangen. Außerdem wurde regelmäßig die Funktionalität getestet, so dass jeder Bestandteil einzeln getestet und abgenommen wurde. 
\subsection{Soll-Ist-Vergleich}
\label{sollIstVgl}
In dem Soll-Ist-Vergleich wird gezeigt, wie schon in \ref{benutzeroberfläche} \nameref{benutzeroberfläche} erwähnt, dass durch die bestehenden Mockups und Absprachen weniger Zeit bei der Entwicklung und dem Entwerfen der Oberfläche benötigt wurde, als geplant. Diese gewonnene Zeit ist allerdings in die wöchentlichen Absprachen zu der Oberfläche, die Analyse bestehender Codebasen, sowie Implementierung der Frontendlogik geflossen.\\

\begin{tabular}{|l|r|r|r|}
\hline
\textbf{Projektphasen }                                   & \textbf{Soll in Std} & \textbf{Ist in Std} & Differenz \\ \hline
Tägliche Absprachen mit Entwicklungsteam \& Auftragsgeber & 1                    & 1                   & 0         \\ \hline
Wöchentliche Absprache zur Oberfläche                     & 0                    & 1                   & +1        \\ \hline
\textbf{Ist-Analyse: }                                    & \textbf{5}           & \textbf{6}          & +1        \\ \hline
Analyse vorhandener Codebasen und Funktionalitäten        & 2                    & 3                   & +1        \\ \hline
Gespräche mit dem Auftragsgeber                           & 3                    & 3                   & 0         \\ \hline
\textbf{Sollkonzept:  }                                   & \textbf{5}           & \textbf{4}          & -1        \\ \hline
Planung des Soll-Zustands Projekt allgemein               & 2                    & 2                   & 0         \\ \hline
Planung des Soll-Zustands Fußzeilenfunktionalität         & 1                    & 1                   & 0         \\ \hline
Abstimmung \& Sichtung Workflows und Oberflächenentwürfe  & 2                    & 1                   & -1        \\ \hline
\textbf{Realisierung: }                                   & \textbf{31}          & \textbf{30}         & -1        \\ \hline
\textbf{Frontend  }                                       & \textbf{10}          & \textbf{12}         & +2        \\ \hline
Entwicklung Oberfläche                                    & 4                    & 1                   & -3        \\ \hline
\textbf{Backend}                                           & 17                   & 17                  & 0         \\ \hline
	\textbf{Dokumentation }                                   & 19                   & 19                  & 0         \\ \hline
	\textbf{Testen \& Abnahme }                               & 8                    & 8                   & 0         \\ \hline
	\textbf{Gesamt}                                           & 69                   & 69                  & 0         \\ \hline
\end{tabular}\\

 Von dem geplanten Umfang des Teilprojekts wurde alles erfüllt. Zu dem wurde die in \ref{abweichung} \nameref{abweichung} erwähnte Angebotskopiespeicherung implementiert.
\subsection{Gewonnene Erkenntnisse}
\label{erkenntnisse}
Die Umsetzung dieses Projekt brachte wertvolle Erkenntnisse. Es wurde das erste Mal mit Vue.js 3 entwickelt, was im Gegensatz zum Vorgänger einige Änderungen mit sich brachte. Dadurch wurde im Bereich der Frontendentwicklung viel neues gelernt. Dazu kommt, dass das erste Mal mit gegebenen Oberflächenmockups gearbeitet wurde, wodurch das Designen und Entwerfen der Oberfläche erleichtert wurde. Außerdem ist durch die Konzeption des Gesamtprojekt von einem Softwarearchitekten eine einheitliche Struktur erkennbar, was das Implementieren neuer Logik und das Anpassen oder Erweitern bestehender Logik einfacher gestaltete.
\subsection{Ausblick}
\label{ausblick}
Als neuste Generation der \ac{OTR} sollen nun alle neuen Tarife anhand dieser Basis implementiert werden. Außerdem ist geplant mindestens einen alten Tarif in die neue Basis zu übertragen.\\
Die Konfigurationen für alle Tarife sind zur Zeit in verschiedenen Systemen verstreut. Geplant ist es diese in einer NoSQL Datenbank mit Anbindung zum neuen \ac{OTR} zu speichern, so dass der \ac{OTR} die zentrale Stelle für Tarifkonfigurationen wird.\\
Für die Fußzeilenkomponente sind noch kleinere Erweiterungen ausstehend, die nicht im Rahmen des Abschlussprojektes umgesetzt werden konnten oder nicht geplant waren. Dazu gehört eine Ladeanimation und Eingabesperre im Frontend, solange auf das Drucken eines Dokuments gewartet wird. Außerdem sind kleinere Refactorings im Backend sowie ein Refactoring der Datenzustandsverwaltung im Frontend geplant.