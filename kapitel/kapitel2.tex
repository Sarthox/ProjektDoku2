\section{Projektplanung}
\label{projektplanung}
Das Projekt wird in dem Zeitraum vom 01.03.2022 bis zum 18.03.2022 durchgeführt. %Wie 3 Wochen = 114 Stunden argumentieren? Zeit auch in Tagesgeschäft? Zeit in umliegende OTR aufgaben die nicht ins Projekt mit reinzählen?

\subsection{Projektphasen}
\label{projektphasen}
%todo
\begin{comment}
todo Lesbarkeit Tabelle: ich schlage vor, nur die Zusammenfassung bzw Summen  jeweils in einer anderen Farbe zu gestalten, so wäre es übersichtlicher, jetzt muss ich erst überlegen und rechnen ob die Summen mit den nächsten Zeilen passen
Realisierung: 30 zb blau als Summe
Frontend 13 zb hellblau als Teilsumme
Knöpfe eingebunden & Styling 3 zb grau oder weiß als Summand
Funktionalität Backend Anbindung, abhängiges Rendering 7
Eventhandling, Texte mit Internationalisierung 2
Entwicklung Oberfläche
\end{comment}


%{\rowcolors{2}{blue!50}{blue!10}
\begin{center}
	\begin{tabular}{|l|r|}
		\hline
		\rowcolor{blue!70}\textbf{Projektphasen }                                  & \textbf{Zeit in Stunden} \\ \hline
		\rowcolor{blue!10}Tägliche Absprachen mit Entwicklungsteam \& Auftraggeber &                        1 \\
		\rowcolor{blue!10}Wöchentliche Absprache zur Oberfläche                    &                        1 \\
		\rowcolor{blue!50}\textbf{Ist-Analyse: }                                   &               \textbf{6} \\
		\rowcolor{blue!10}Analyse vorhandener Codebasen und Funktionalitäten       &                        3 \\
		\rowcolor{blue!10}Gespräche mit dem Auftragsgeber                          &                        3 \\
		\rowcolor{blue!50}\textbf{Sollkonzept:  }                                  &               \textbf{4} \\
		\rowcolor{blue!10}Planung des Soll-Zustands Projekt allgemein              &                        2 \\
		\rowcolor{blue!10}Planung des Soll-Zustands Fußzeilenfunktionalität        &                        1 \\
		\rowcolor{blue!10}Abstimmung \& Sichtung Workflows und Oberflächenentwürfe &                        1 \\
		\rowcolor{blue!50}\textbf{Realisierung: }                                  &              \textbf{30} \\
		\rowcolor{blue!30}\textbf{Frontend  }                                      &              \textbf{13} \\
		\rowcolor{blue!10}Schaltflächeneinbindung \& Styling                     &                        3 \\
		\rowcolor{blue!10}Funktionalität Backend Anbindung, abhängiges Rendering   &                        7 \\
		\rowcolor{blue!10}Eventhandling, Texte mit Internationalisierung           &                        2 \\
		\rowcolor{blue!10}Entwicklung Oberfläche                                   &                        1 \\
		\rowcolor{blue!30}\textbf{Backend  }                                       &             \textbf{ 17} \\
		\rowcolor{blue!10}Angebotsspeicherung                                      &                        8 \\
		\rowcolor{blue!10}Angebotskopiespeicherung                                 &                        2 \\
		\rowcolor{blue!10}Dokumentengenerierung                                    &                        5 \\
		\rowcolor{blue!10}Dokumentenversand                                        &                        2 \\
		\rowcolor{blue!50}\textbf{Dokumentation }                                  &             \textbf{ 19} \\
		\rowcolor{blue!50}\textbf{Testen \& Abnahme }                              &              \textbf{ 8} \\ \hline
		\rowcolor{blue!70}\textbf{Gesamt}                                          &             \textbf{ 69} \\ \hline
	\end{tabular}
\end{center}
\captionof{table}{Gebrauchte Zeit in Stunden aufgeteilt in Projektphasen}

\subsection{Abweichung vom Projektantrag}
\label{abweichung}
Alle im Projektantrag beschriebenen Funktionalitäten werden umgesetzt. Hinzu kommt die Kopiespeicherung eines Angebots. Außerdem ist schon ein Grundgerüst in Form von bestehenden Serviceprojekten und Schnittstellen gegeben. Die UI-Elemente müssen teilweise neu erstellt, erweitert oder angepasst werden.\\
Dadurch, dass Mockups für die Benutzeroberfläche gegeben sind, muss weniger Zeit von meiner Seite aus in die Konzeption der Benutzeroberfläche fließen. Deswegen weichen die angegebenen Zeiten aus Abschnitt \ref{projektphasen} \nameref{projektphasen} leicht von den Zeiten aus dem Projektantrag ab (s. \ref{sollIstVgl} \nameref{sollIstVgl}). 
\subsection{Ressourcenplanung}
\label{ressourcenplanung}
Für die Umsetzung des Projekts wurden Ressourcen verwendet für die \ac{KM} bereits Lizenzen hat oder die unentgeltlich zur Verfügung stehen. Des Weiteren wurde Hardware verwendet die bereits im Besitz des Unternehmens ist.
Folgende Ressourcen wurden verwendet:

\begin{itemize}
	\item \textbf{Technische Ressourcen}
	\begin{itemize}
		\item Windows 10
		\item Visual Studio 2022
		\item Webstorm 2021
		\item TeXstudio
		\item GitLab
		\item Fork - GitClient
		\item Kubernetes/Docker
		\item Microsoft Teams für Besprechungen
	\end{itemize}
	\item \textbf{Personen}
	\begin{itemize}
	 	\item Produktmanagement, für Rückfragen zum umzusetzenden Produkt
	 	\item Vertrieb
	 	\item externer Versicherer für Anforderungsdefinition
		\item Marketingabteilung für Designanforderungen
		\item Auszubildender für die Umsetzung und Tests
	\end{itemize}
\end{itemize}

\subsection{Entwicklungsprozess}
\label{entwicklungsprozess}
%Ausdenken welches genau benutzt wurde? André fragen wieso?
Das Projekt wird mit dem Spiralmodell durchgeführt, unterstützt durch ein Kanban-Board. Vor dem Start der Bearbeitung wurden grobe Aufgaben und Anforderungen definiert. Es gibt tägliche Absprachen zwischen dem Entwicklerteam sowie wöchentliche Absprachen mit dem Produktmanagement und der Marketingabteilung bezüglich der Oberfläche. Aus diesen Absprachen entstehen iterativ neue Anforderungen. Durch \ac{CICD} werden inkrementell umgesetzte Funktionalitäten bzw. behobene Fehler zum Testen zur Verfügung gestellt. \\
Dieses Vorgehensmodell wurde gewählt damit man die Entwicklung schnell an neue Anforderungen der Kunden bzw. der Stakeholder anpassen kann.