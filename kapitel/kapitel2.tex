\newpage
\section{Projektplanung}
\label{projektplanung}
Das Projekt wird in dem Zeitraum vom 01.03.2022 \marginpar{Achtung, richtiges Datum} bis einschließlich den 18.03.2022 durchgeführt. %Wie 3 Wochen = 114 Stunden argumentieren? Zeit auch in Tagesgeschäft? Zeit in umliegende OTR aufgaben die nicht ins Projekt mit reinzählen?

\subsection{Projektphasen}
\label{projektphasen}

{\rowcolors{2}{blue!50}{blue!10}
\begin{tabular}{|l|r|}
	\hline
	\textbf{Projektphasen }                                  & \textbf{Zeit in Stunden} \\ \hline
	tägliche Absprachen mit Entwicklungsteam + Auftragsgeber &                        1 \\
	wöchentliche Absprache zur Oberfläche                    &                        2 \\
	\textbf{Ist-Analyse: }                                   &                        5 \\
	Analyse vorhandener Codebasen und Funktionalitäten       &                        3 \\
	Gespräche Auftragsgeber                                  &                        4 \\
	\textbf{Sollkonzept:  }                                  &                        5 \\
	Planung des Soll-Zustands Projekt allgemein              &                        2 \\
	Planung des Soll-Zustands Fußzeilenfunktionalität        &                        1 \\
	Abstimmung + Sichtung Workflows und Oberflächenentwürfe  &                        1 \\
	\textbf{Realisierung: }                                  &                       31 \\
	\textbf{Frontend  }                                      &                       10 \\
	Knöpfe eingebunden + Styling                             &                        2 \\
	Funktionalität Backend Anbindung, optionales Rendering   &                        6 \\
	Eventhandling, Label mit i18n                            &                        2 \\
	\textbf{Backend  }                                       &                       17 \\
	Angebotspeicherung                                       &                        8 \\
	Kopiespeicherung                                         &                        2 \\
	Dokumentengenerierung                                    &                        7 \\
	Entwicklung Oberfläche                                   &                        1 \\
	\textbf{Dokumentation }                                  &                       19 \\
	Testen + Abnahme                                         &                        8 \\ \hline
	\textbf{	Gesamt    }                                  &                       69 \\ \hline
\end{tabular}


\subsection{Abweichung vom Projektantrag}
\label{abweichung}
Alle im Projektantrag beschriebenen Funktionalitäten wurden umgesetzt. Hinzu kam die Kopiespeicherung eines Angebotes. Außerdem war schon mehr Grundgerüst im Form von bestehenden Serviceprojekten und Schnittstellen gegeben, wodurch nur noch die reine Logik implementiert werden musste. Die UI-Elemente mussten neu erstellt werden oder ggf. erweitert oder angepasst werden.
\subsection{Ressourcenplanung}
\label{ressourcenplanung}
Für die Umsetzung des Projekts wurden Ressourcen verwendet für die \ac{KM} bereits Lizenzen hat oder die unentgeltlich zur Verfügung stehen. Des Weiteren wurde Hardware verwendet die bereits im Besitz des Unternehmens ist.
Folgende Ressourcen wurden verwendet:

\begin{itemize}
	\item \textbf{Technische Ressourcen}
	\begin{itemize}
		\item Windows 10
		\item Visual Studio 2022
		\item Webstorm 2021
		\item TeXstudio
		\item GitLab
		\item Fork - GitClient
		\item Kubernetes/Docker
		\item Microsoft Teams für Besprechungen
	\end{itemize}
	\item \textbf{Personen}
	\begin{itemize}
	 	\item Produktmanagement, Vertrieb, externer Versicherer für Anforderungsdefinition
		\item Marketingabteilung für Designanforderungen
		\item Auszubildender für die Umsetzung und Tests
	\end{itemize}
\end{itemize}

\subsection{Entwicklungsprozess}
\label{entwicklungsprozess}
Scrum??? => spiral + kanban
