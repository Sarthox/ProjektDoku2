\section{Projektplanung}
\label{projektplanung}
Das Projekt wird in dem Zeitraum vom 01.03.2022 bis zum 19.03.2022 durchgeführt. %Wie 3 Wochen = 114 Stunden argumentieren? Zeit auch in Tagesgeschäft? Zeit in umliegende OTR aufgaben die nicht ins Projekt mit reinzählen?

\subsection{Projektphasen}
\label{projektphasen}

{\rowcolors{2}{blue!50}{blue!10}
\begin{tabular}{|l|r|}
	\hline
	\textbf{Projektphasen }                                   & \textbf{Zeit in Stunden} \\ \hline
	Tägliche Absprachen mit Entwicklungsteam \& Auftragsgeber &                        1 \\
	Wöchentliche Absprache zur Oberfläche                     &                        1 \\
	\textbf{Ist-Analyse: }                                    &               \textbf{6} \\
	Analyse vorhandener Codebasen und Funktionalitäten        &                        3 \\
	Gespräche mit dem Auftragsgeber                           &                        3 \\
	\textbf{Sollkonzept:  }                                   &               \textbf{4} \\
	Planung des Soll-Zustands Projekt allgemein               &                        2 \\
	Planung des Soll-Zustands Fußzeilenfunktionalität         &                        1 \\
	Abstimmung \& Sichtung Workflows und Oberflächenentwürfe  &                        1 \\
	\textbf{Realisierung: }                                   &              \textbf{30} \\
	\textbf{Frontend  }                                       &              \textbf{13} \\
	Knöpfe eingebunden \& Styling                             &                        3 \\
	Funktionalität Backend Anbindung, abhängiges Rendering    &                        7 \\
	Eventhandling, Texte mit Internationalisierung            &                        2 \\
		Entwicklung Oberfläche                                    &                        1 \\
	\textbf{Backend  }                                        &             \textbf{ 17} \\
	Angebotsspeicherung                                       &                        8 \\
	Angebotskopiespeicherung                                  &                        2 \\
	Dokumentengenerierung                                     &                        7 \\
	\textbf{Dokumentation }                                   &             \textbf{ 19} \\
	\textbf{Testen \& Abnahme }                               &              \textbf{ 8} \\ \hline
	\textbf{	Gesamt    }                                      &             \textbf{ 69} \\ \hline
\end{tabular}


\subsection{Abweichung vom Projektantrag}
\label{abweichung}
Alle im Projektantrag beschriebenen Funktionalitäten wurden umgesetzt. Hinzu kam die Kopiespeicherung eines Angebotes. Außerdem war schon mehr Grundgerüst im Form von bestehenden Serviceprojekten und Schnittstellen gegeben, wodurch die reine Logik implementiert werden musste. Die UI-Elemente mussten teilweise neu erstellt, erweitert oder angepasst werden.\\
Dadurch, dass Mockups für die Benutzeroberfläche gegeben sind, weichen die angegebenen Zeiten aus Abschnitt \ref{projektphasen} leicht von den Zeiten aus dem Projektantrag ab.
\subsection{Ressourcenplanung}
\label{ressourcenplanung}
Für die Umsetzung des Projekts wurden Ressourcen verwendet für die \ac{KM} bereits Lizenzen hat oder die unentgeltlich zur Verfügung stehen. Des Weiteren wurde Hardware verwendet die bereits im Besitz des Unternehmens ist.
Folgende Ressourcen wurden verwendet:

\begin{itemize}
	\item \textbf{Technische Ressourcen}
	\begin{itemize}
		\item Windows 10
		\item Visual Studio 2022
		\item Webstorm 2021
		\item TeXstudio
		\item GitLab
		\item Fork - GitClient
		\item Kubernetes/Docker
		\item Microsoft Teams für Besprechungen
	\end{itemize}
	\item \textbf{Personen}
	\begin{itemize}
	 	\item Produktmanagement, für Rückfragen zum umzusetzendem Produkt
	 	\item Vertrieb
	 	\item externer Versicherer für Anforderungsdefinition
		\item Marketingabteilung für Designanforderungen
		\item Auszubildender für die Umsetzung und Tests
	\end{itemize}
\end{itemize}

\subsection{Entwicklungsprozess}
\label{entwicklungsprozess}
%Ausdenken welches genau benutzt wurde? André fragen wieso?
Das Projekt wurde mit dem Spiralmodell durchgeführt, unterstützt durch ein Kanban-Board. Vor dem Start der Bearbeitung wurden grobe Aufgaben und Anforderungen definiert. Es gab tägliche Absprachen zwischen dem Entwicklerteam sowie wöchentliche Absprachen mit dem Produktmanagement und der Marketingabteilung bezüglich der Oberfläche. Aus diesen Absprachen sind iterativ neue Anforderungen entstanden. Durch \ac{CICD} wurden inkrementell umgesetzte Funktionalitäten bzw. behobene Fehler zum Testen zur Verfügung gestellt. \\
Dieses Vorgehensmodell wurde gewählt damit man die Entwicklung schnell an neue Anforderung der Kunden bzw. der Stakeholder anpassen kann.