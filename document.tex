% Allgemeines Dokumentenformnat
\documentclass[a4paper,11pt,headsepline,hidelinks]{scrartcl}
\usepackage{acronym}
\usepackage{comment}
\usepackage[table, xcdraw]{xcolor}
\usepackage[
automark, % Kapitelangaben in Kopfzeile automatisch erstellen
headsepline, % Trennlinie unter Kopfzeile
ilines % Trennlinie linksbündig ausrichten
]{scrlayer-scrpage}
% Kopf- und Fußzeilen ----------------------------------------------------------
\pagestyle{scrheadings}
% chapterpagestyle gibt es nicht in scrartcl
%\renewcommand{\chapterpagestyle}{scrheadings}
\clearscrheadfoot

% Kopfzeile 
\renewcommand{\headfont}{\normalfont} % Schriftform der Kopfzeile
%\ihead{\large{\textsc{Anwendung zur Erfassung und Bearbeitung von\\ umfangreichen Anwender-Softwaretests}}\\ \small{Softwaretest Manager} \\[0ex]}
%\chead{}
\ohead{\includegraphics[scale=0.2]{bilder/k&m.png}}
%\setlength{\headheight}{15mm} % Höhe der Kopfzeile
%\setheadwidth[0pt]{textwithmarginpar} % Kopfzeile über den Text hinaus verbreitern (falls Logo den Text überdeckt)

% Fußzeile
\ifoot{Jakob Schumann}
\cfoot{}
\ofoot{\pagemark}

% Überschriften nach DIN 5008 in einer Fluchtlinie
% ------------------------------------------------------------------------------

% Abstand zwischen Nummerierung und Überschrift definieren
% > Schön wäre hier die dynamische Berechnung des Abstandes in Abhängigkeit
% > der Verschachtelungstiefe des Inhaltsverzeichnisses
\newcommand{\headingSpace}{1.5cm}

% Abschnittsüberschriften im selben Stil wie beim Inhaltsverzeichnis einrücken
\renewcommand*{\othersectionlevelsformat}[3]{
	\makebox[\headingSpace][l]{#3\autodot}
}

% Für die Einrückung wird das Paket tocloft benötigt
%\cftsetindents{chapter}{0.0cm}{\headingSpace}
%\cftsetindents{section}{0.0cm}{\headingSpace}
%\cftsetindents{subsection}{0.0cm}{\headingSpace}
%\cftsetindents{subsubsection}{0.0cm}{\headingSpace}
%\cftsetindents{figure}{0.0cm}{\headingSpace}
%\cftsetindents{table}{0.0cm}{\headingSpace}
% Weitere Pakete
% Grafiken
\usepackage{graphicx}
\usepackage{float}

% Deutsche Sonderzeichen
 %\usepackage{ngerman}

% Deutsche Silbentrennung
\usepackage[ngerman]{babel} 

% Eurozeichen einbinden
\usepackage[right]{eurosym} 

% Umlaute in UTF8
\usepackage[utf8]{inputenc}

% Zeichencodierung
\usepackage[T1]{fontenc}
\usepackage{mathptmx}
\usepackage{helvet} 
%\usepackage{fix-cm}
%\newcommand{\changefont}[3]{	\fontfamily{#1} \fontseries{#2} \fontshape{#3} \selectfont}
%\changefont{ptm}{m}{sl}

% mehrseitige Tabellen
\usepackage{longtable}

% Seitenränder
\usepackage{geometry}
\geometry{left=2.5cm, right=2cm, top=4cm, bottom=2cm}

% Boxen im Text
\usepackage{fancybox}

% schöne URLs
\usepackage[hyphens,obeyspaces,spaces]{url} % schöne URLs

% Textfarben
\usepackage{color}

% Mathematische Symbole
\usepackage{amssymb}

% Inhaltsverzeichnis + Querverweiss
\usepackage[bookmarksnumbered,pdftitle={Abschlussprojektdokumentation},hyperfootnotes=false]{hyperref}

\hypersetup{
	colorlinks=false,
	linkcolor=blue,
	filecolor=magenta,
	urlcolor=cyan,
}

% Kopfzeile
%\usepackage{fancyhdr}
%\pagestyle{fancy}
%\fancyhf{}
%\fancyhead[L]{\nouppercase{\leftmark}}
%\fancyhead[C]{}
%\fancyhead[R]{\thepage}
%\renewcommand{\headrulewidth}{0.4pt}

% Tabellen allgemein
\usepackage{array}

% Zitierweise
\bibliographystyle{alphadin}

% Kein Zwischenraum nach Satzzeichen
\frenchspacing

% Zeilenabstand
\usepackage{setspace}
\usepackage{parskip}
\parskip 6pt plus 3pt minus 1pt

% Bildzeichner
\usepackage{capt-of}
\usepackage{caption}

% Stichwortverzeichnis
\usepackage{makeidx}
%\usepackage{unnumberedtotoc}

% Aufzählungen
\usepackage{listings}
\lstset{numbers=left, numberstyle=\tiny, numbersep=5pt, keywordstyle=\color{black}\bfseries, stringstyle=\ttfamily,showstringspaces=false,basicstyle=\footnotesize,captionpos=b}
\lstset{language=bash}

% Indexerstellubg
\makeindex

% Glossar
\usepackage{glossaries}
\makeglossaries
\newglossaryentry{k8s}{name={Kubernetes}, description={Orchestrator für Containeranwendungen}, text={\underline{Kubernetes}}}
\newglossaryentry{vue}{name={Vue.js}, description={Ein clientseitiges Javascript-Framework nach dem Model-View-ViewModel-Entwurfsmuster für das Erstellen von Webanwendungen}, text={\underline{Vue.js}}}
\newglossaryentry{vuetify}{name={Vuetify}, description={Eine Vue.js Oberflächenkomponentenbibliothek}, text={\underline{Vuetify}}}
\newglossaryentry{typescript}{name={TypeScript}, description={Eine typsichere Erweiterung von JavaScript}, text={\underline{TypeScript}}}
\newglossaryentry{npm}{name={npm}, description={Ein Paketmanager für Node.js Pakete}, text={\underline{npm}}}
\newglossaryentry{pinia}{name={Pinia-Store}, description={Eine Datenzustandsverwaltung für Vue.js welche den Localstorage des Browsers verwendet}, text={\underline{Pinia-Store}}}
\newglossaryentry{HTTP}{name={HTTP}, description={Hypertext Transfer Protocol; Protokoll für Datenübertragung}, text={\underline{HTTP}}}
\newglossaryentry{fetch}{name={Javascript-fetch}, description={Asynchroner Aufruf an ein Backend}, text={\underline{JavaScript-Fetch}}}
\newglossaryentry{HTML}{name={HTML}, description={Hypertext Markup Language; textbasierte Auszeichnungssprache um Inhalte in Webbrowsern darzustellen}, text={\underline{HTML}}}
\newglossaryentry{Container}{name={Container}, description={Laufende Instanz eines Softwarepakets im Dockerumfeld}, text={\underline{Container}}}
%\newglossaryentry{helm}{name={Helm}, description={Packetmanager und Hilfstool für Kubernetes}, text={\underline{Helm}}}
\newglossaryentry{repo}{name={Repository}, description={Verzeichnis zur versionierten Speicherung von Quellcode}, text={\underline{Repository}}}
\newglossaryentry{repopattern}{name={Repository-Pattern}, description={Ein Designpattern, welches die Logik der Datenspeicherung und -beschaffung kapselt, so dass eine zentrale Stelle für Datenquellen besteht}, text={\underline{Repository-Pattern}}}
\newglossaryentry{ddd}{name={Domain-Driven Design}, description={Modellierung der Software nach umzusetzenden Fachlichkeiten der Anwendungsdomäne}, text={\underline{Domain-Driven Design}}}
\newglossaryentry{REST}{name={REST}, description={Reprensentational State Transfer; Basierend auf HTTP; ein Paradigma zur Kommunikation zwischen Systemen}, text={\underline{REST}}}
\newglossaryentry{Linter}{name={Linter}, description={Ein Tool zur statischen Codeanalyse}, text={\underline{Linter}}}
%\newglossaryentry{automapper}{name={Automapper}, description={Eine Bibliothek mit der eine Objekt-zu-Objekt Datenübertragungsstrategie konfiguriert werden kann} , text={\underline{Automapper}}}
\newglossaryentry{net}{name={.NET}, description={Eine Softwareplattform für das Entwickeln und Ausführen von Anwendungen}, text={\underline{.NET}}}
\newglossaryentry{api}{name={API}, description={Application Programming Interface, Programmierschnittstelle; ein Programmteil der zur Anbindung für andere Systeme zur Verfügung gestellt wird}, text={\underline{API}}}
\newglossaryentry{di}{name={Dependency Injection}, description={Ein Entwurfsmuster welches Abhängigkeiten von Objekten zur Laufzeit auflöst}, text={\underline{Dependency Injection}}}
\newglossaryentry{localstorage}{name={Localstorage}, description={Ein persistenter clientseitiger Datenspeicher im Webbrowser}, text={\underline{Localstorage}}}
\newglossaryentry{git}{name={Git}, description={Dezentrale Versionverwaltung}, text={\underline{Git}}}
\newglossaryentry{gitflow}{name={GitFlow}, description={Ein Modell für das Benutzen von Gitbranches, in welchem eigenständige Anforderungen in eigene Branches aufgeteilt werden}, text={\underline{GitFlow}}}
\newglossaryentry{gitlab}{name={GitLab}, description={Ein Anbieter von Git mit DevOps-Funktionen}, text={\underline{GitLab}}}
\newglossaryentry{NSwag}{name={NSwag}, description={Ein Tool um Swagger OpenApi Beschreibungen zu generieren}, text={\underline{NSwag}}}
\newglossaryentry{distributedcache}{name={Distributed Cache}, description={Verteilter Cache, Speicher der sich über mehrere Instanzen spannt}, text={\underline{Distributed Cache}}}
\newglossaryentry{php}{name={PHP}, description={Eine Skriptsprache die zur Erstellung dynamischer Webseiten dient}, text={\underline{PHP}}}

\begin{document}

% Titelseite
\thispagestyle{empty}


\begin{center}
	\includegraphics[scale=0.9]{bilder/logo_ihkdata}\\[2ex]
	\Large{Abschlussprüfung Sommer 2022}\\[3ex]
	
	\Large{Fachinformatiker für Anwendungsentwicklung}\\
	\LARGE{Dokumentation zur betrieblichen Projektarbeit}\\[4ex]
	
	\huge{\textbf{Fußzeile als Komponente mit funktionalen Knöpfen und Darstellung von Informationen für die Abschlussstrecke eines Versicherungsproduktes}}\\[1ex]
	
	\normalsize
	Abgabetermin: Hannover, den 27.04.2022\\
	Prüfungsausschuss: FIAE 3\\[3em]
	
	\textbf{Prüfungsbewerber:}\\
	\begin{onehalfspace}
	Jakob Schumann\\
	Goethestraße 40\\
	30169 Hannover\\[5ex]
	\end{onehalfspace}
	 
%	\includegraphics[scale=0.8] {}\\[1ex]

	\textbf{Ausbildungsbetrieb:}\\
	\begin{onehalfspace}
	Konzept \& Marketing – ihr unabhängiger Konzeptentwickler GmbH\\
	Podbielskistraße 333\\
	30659 Hannover\\[5em]
	\end{onehalfspace}
\end{center}


\pagenumbering{Roman}
\onehalfspacing


% Inhaltsverzeichnis
\tableofcontents
\thispagestyle{empty}
\listoftables
\listoffigures
\thispagestyle{empty}


%Abstract
%\clearpage
%\addsec{Abstract}

% Tabellenverzeichnis
%\clearpage


% Abkürzungsverzeichnis
%\clearpage

% Kapitel
\clearpage
\pagenumbering{arabic}
%\fancyhead[L]{\nouppercase{\leftmark}}

% Einleitung

\section{Einleitung}
\label{einleitung}

\subsection{Projektumfeld}
\label{projektumfeld}
\textbf{\acl{KM}}
\\

Bei der \ac{KM} handelt es sich um einen Finanzdienstleister der Versicherungskonzepte entwickelt. Die Konzepte werden über Partnerschaften mit namhaften Versicherern auf den Markt gebracht. \ac{KM} übernimmt die vollständige Verwaltung, vom Vertrieb über freie Makler über die Vertragsannahme, bis zur Schadenregulierung.
Das Unternehmen, bestehend derzeit aus 118 Mitarbeitern, bietet Dienstleistungen sowie Portale für Versicherer, Makler und Kunden an.

\textbf{Mariosoft}
\\

Bei der Mariosoft handelt es sich um eine IT-Abteilung der \ac{KM}. Die Mariosoft besteht derzeit aus 18 Mitarbeitern und beschäftigt sich mit der Konzeption, Programmierung und Wartung von internen Softwarelösungen wie \ac{OTR}, Kundenverwaltung und Übersichtsseiten für Makler.
\\
Das umzusetzende Projekt ist im Umfeld der Mariosoft angesiedelt und ist eine Komponente für eine gleichzeitig entstehende Webanwendung.

\subsection{Projektziel}
\label{projektziel}
Ziel des Gesamtprojekts ist ein neuer, modularer \ac{OTR} als Webseite bzw. \ac{SPA}. Dieser soll so entworfen werden, dass die einzelnen Komponenten austauschbar sind und für weitere Versicherungsprodukte benutzt werden können. Außerdem soll die Benutzeroberfläche einheitlich gestaltet werden. Allgemein soll die Einführung neuer und die Erweiterung bestehender Versicherungsprodukte vereinfacht werden. Der \ac{OTR} wird bestehen aus einem \gls{net} 6.0 Backend, welches Schnittstellen zu umliegenden firmeninternen Systemen hat. Außerdem ist ein Cachespeicher und weitere Logik vorhanden. Das Frontend besteht aus \gls{vue} mit Designvorlagen von \gls{vuetify} sowie einer eigenen Oberflächenkomponentenbibliothek, welche \gls{vuetify}elemente kapselt und im Corporatedesign zur Verfügung stellt. Im Frontend wird \gls{typescript} als Programmiersprache benutzt.\\
Als Versionsverwaltung für dieses Projekt wird \gls{git} mit \gls{gitflow} genutzt. Das \ac{CICD} wird mittels \gls{gitlab} Pipelines realisiert, also eine Möglichkeit automatisch nach jeder Änderung das Projekt zu bauen, zu testen und zu veröffentlichen. Veröffentlicht wird der \ac{OTR} in einem \gls{k8s}-Cluster. Im Abschnitt \ref{zielplattform} wird auf das Verfahren weiter eingegangen.
\\
Bei einem \ac{OTR} hat ein Makler die Möglichkeit Beiträge für versicherte Risiken zu errechnen. Außerdem sind hilfreiche oder sogar verpflichtende Funktionalitäten gegeben, wie z.B. das Ausdrucken von Anträgen und Angeboten, das Speichern von Angeboten, um diese später wieder laden und weiter bearbeiten zu können und das Versenden von tarifrelevanten Dokumenten per E-Mail. Diese Funktionalitäten werden in der Fußzeile in Form von Schaltflächen angeboten. Die Implementierung dieser Fußzeile wird von mir im Rahmen dieses Projekts umgesetzt. Die Fußzeile beinhaltet außerdem Angaben zum Copyright und einen Verweis zum Impressum.
\\

%Angebotsspeicherung
Beim Speichern eines Angebots sollen alle im \ac{OTR} eingegebenen Daten im sogenannten Maklerportal für den angemeldeten Makler gespeichert werden. Diese können über das Maklerportal wieder geladen werden. Außerdem ist es vorgesehen, dass Makler Angebote kopieren können.
Die \ac{KM} bietet auch \ac{OTR} außerhalb des Maklerportals an. Für diese Rechner ist dann keine Angebots- oder Kopiespeicherung vorgesehen, dementsprechend sind die Schaltflächen dort nicht vorhanden. \\

% Drucken
Die Druckenfunktionalität wird durch zwei Schaltflächen abgebildet, eine um ein Angebot zu drucken, die zweite für den Antrag. Ein Angebot wird z.B. dem Endkunden vom Makler unterbreitet und ist, genau wie der Antrag, ein verbindliches Dokument mit welchem ein Versicherungsvertrag abgeschlossen werden kann. Die Funktionalität wird häufig genutzt und ist damit wichtig. Die Erzeugung von Angebot- und Antragsdokumenten setzt die Berechnung der Risiken voraus. Dafür müssen alle berechnungsrelevanten Angaben getroffen worden sein. Nur bei einer gültigen Berechnung werden die Schaltflächen aktiv. Außerdem kann es sein, dass für einen Tarif der Angebots- oder Antragsdruck deaktiviert ist. Dementsprechend sind die Schaltflächen aktiviert oder nicht aktiviert. Das Drucken ist dabei eigentlich das Herunterladen eines PDF-Dokumentes. Das Dokument wird von einem angebundenem System anhand der Daten in XML-Format generiert.

%Versand
Eine weitere Funktion ist die des Dokumentenversands. Bei Drücken dieser Schaltfläche öffnet sich ein Dialog, wo der Nutzer die gewünschten Dokumente aus allen relevanten Dokumenten auswählen kann um sie zu versenden. Dazu gehören auch die zuvor beschriebenen Angebots- und Antragsdokumente. Dazu muss noch eine oder mehrere E-Mailadressen angegeben werden und dann werden die Dokumente an diese E-Mailadressen in einem ZIP-Archiv versendet. Die Dokumente beinhalten rechtliche Informationen und den Antrag sowie das Angebot. Die komplette Dokumentenversand-Komponente gibt es als eigenständige Bibliothek, da sie auch für andere \ac{OTR} genutzt wird, so dass diese für das Frontend eingebunden werden musste.

\subsection{Projektbegründung}
\label{projektbegründung}
Die Motivation hinter diesem Projekt begründet sich in der einfachen Erweiterbarkeit für neue Versicherungsprodukte sowie ein Redesign des Aussehens des \ac{OTR}. Der \ac{OTR} ist durch den modularen und komponentenbasierten Aufbau des Projekts besser wartbar. \\
Die Fußzeile ist eine Komponente die in allen bestehenden Produkten und zukünftig kommenden Produkten vorhanden sein muss. Die Platzierung der gegebenen Funktionalitäten in die Fußzeile bietet sich dafür gut an, da sie unabhängig von der übrigen Webseite funktionieren soll.
\subsection{Projektschnittstellen}
\label{projektschnittstellen}
Das Backend hat Schnittstellen zu einigen Systemen. Es gibt das \ac{TG}, welches den Einstiegspunkt von Extern zu den \ac{OTR} darstellt. Dieses sowie der \ac{OTR} benutzt das \ac{TRG}, welches als REST-Schnittstelle funktioniert um unter anderem Konfigurationen für Tarife oder das Speichern von Angeboten anzubieten.
Eine weitere Schnittstelle die von der Fußzeilenkomponente genutzt wird ist der PDFToolsservice, welcher die Generierung von PDF-Dokumenten übernimmt. Die Fußzeile sowie alle Komponenten im Frontend bekommen ihre Daten über den \gls{localstorage} des Browsers (\gls{vue} \gls{pinia}) geliefert.\\
Das Maklerportal ist eine Webseite an der sich Makler anmelden und den \ac{OTR} aufrufen können, Übersichten über ihren Kundenbestand haben und noch vieles mehr. Außerdem steuert das Maklerportal die Konfiguration der \ac{OTR}. Zudem können dort auch gespeicherte Angebote eingesehen, bearbeitet und mit einem \ac{OTR} aufgerufen werden.\\
Der \ac{OTR} wird vor allem von Maklern benutzt um Tarife für ihre Kunden auszurechnen.


\subsection{Projektabgrenzung}
\label{projektabgrenzung}
Das bearbeitete Projekt ist hierbei die Fußzeile mit den Schaltflächen einschließlich ihrer Funktionalitäten. Eine Benutzeroberflächenkomponentenbibliothek ist gegeben, aus welcher die Schaltflächen eingebunden werden. Genauso ist eine Synchronisation zwischen Front- und Backend durch HTTP-JSON-PATCH schon vorhanden. Weiterhin ist ein Backend mit mehreren Schnittstellen welche nur für die einzelnen Funktionalitäten erweitert werden müssen verfügbar. Im Frontend ist ein Gridlayout mit Elementvorlagen gegeben, welche genauso erweitert werden müssen.
\newpage
\section{Projektplanung}
\label{projektplanung}
Das Projekt wird in dem Zeitraum vom 21.02.2022 \marginpar{Achtung, richtiges Datum} bis einschließlich den 22.04.2022 durchgeführt.

\subsection{Projektphasen}
\label{projektphasen}

{\rowcolors{2}{blue!50}{blue!10}
\begin{tabular}{|l|r|}
	\hline
	\textbf{Projektphasen }                                  & \textbf{Zeit in Stunden} \\ \hline
	tägliche Absprachen mit Entwicklungsteam + Auftragsgeber &                        1 \\
	wöchentliche Absprache zur Oberfläche                    &                        2 \\
	\textbf{Ist-Analyse: }                                   &                        5 \\
	Analyse vorhandener Codebasen und Funktionalitäten       &                        3 \\
	Gespräche Auftragsgeber                                  &                        4 \\
	\textbf{Sollkonzept:  }                                  &                        5 \\
	Planung des Soll-Zustands Projekt allgemein              &                        2 \\
	Planung des Soll-Zustands Fußzeilenfunktionalität        &                        1 \\
	Abstimmung + Sichtung Workflows und Oberflächenentwürfe  &                        1 \\
	\textbf{Realisierung: }                                  &                       31 \\
	\textbf{Frontend  }                                      &                       10 \\
	Knöpfe eingebunden + Styling                             &                        2 \\
	Funktionalität Backend Anbindung, optionales Rendering   &                        6 \\
	Eventhandling, Label mit i18n                            &                        2 \\
	\textbf{Backend  }                                       &                       17 \\
	Angebotspeicherung                                       &                        8 \\
	Kopiespeicherung                                         &                        2 \\
	Dokumentengenerierung                                    &                        7 \\
	Entwicklung Oberfläche                                   &                        1 \\
	\textbf{Dokumentation }                                  &                       19 \\
	Testen + Abnahme                                         &                        8 \\ \hline
	\textbf{	Gesamt    }                                  &                       69 \\ \hline
\end{tabular}


\subsection{Abweichung vom Projektantrag}
\label{abweichung}
Alle im Projektantrag beschriebenen Funktionalitäten wurden umgesetzt. Hinzu kam die Kopiespeicherung eines Angebotes. Außerdem war schon mehr Grundgerüst im Form von bestehenden Serviceprojekten und Schnittstellen gegeben, wodurch nur noch die reine Logik implementiert werden musste.
\subsection{Ressourcenplanung}
\label{ressourcenplanung}
Für die Umsetzung des Projekts wurden Ressourcen verwendet für die \ac{KM} bereits Lizenzen hat oder die unentgeltlich zur Verfügung stehen. Des Weiteren wurde Hardware verwendet die bereits im Besitz des Unternehmens ist.
Für die Umsetzung des Projekts wurden folgende Ressourcen verwendet:
\begin{itemize}
\item Windows 10
\item Visual Studio 2022
\item Webstorm 2021
\item TeXstudio
\item GitLab
\item Kubernetes/Docker
\item Microsoft Teams für Besprechungen
\item Produktmanagement, Vertrieb, externer Versicherer für Anforderungsdefinition
\item Marketingabteilung für Designanforderungen
\item Auszubildender für die Umsetzung und Tests
\end{itemize}

\subsection{Entwicklungsprozess}
\label{entwicklungsprozess}
Scrum??? => spiral + kanban

\newpage
\section{Analysephase}
\label{analysephase}

\subsection{Ist-Analyse}
\label{ist}
\begin{comment}
Wie eingangs in Kapitel \ref{projektumfeld} kurz erwähnt, verwendet \ac{SAZ} ein \ac{ERP}-System, welches regelmäßige Updates erhält und individuell angepasst werden kann. Durch die individuellen Anpassungen muss nach jedem Update, vor Produktivsetzung, ein Test durch die Anwender erfolgen. Die Resonanz dieser Testläufe, sowie die nachgelagerte Kommunikation aus Emails und ggf. Telefonaten, werden zurzeit über ein Exchange-Postfach verwaltet. Die eingehenden Rückmeldungen werden von den Entwicklern zu Aufgaben formuliert, anschließend kategorisiert, kommentiert und in einer Excel-Datei gelistet, welche als Grundlage zur weiteren Bearbeitung und Lösung der Problemstellungen dient.  Die Kategorisierung erfolgt auf Basis der Dringlichkeit. Tasks, die in der Produktivsetzung die Arbeit einschränken würden, werden vorrangig bearbeitet. Die Excel-Datei erhält zusätzlich eine farbliche Kategorisierung, sodass die Entwickler in kurzer Zeit einen Überblick erhalten, welche Tasks abgeschlossen, behoben, in Bearbeitung sind oder mit dem \ac{ERP}-Hersteller besprochen werden müssen. Die Liste beinhaltet noch weitere Informationen. Entwickler können für die einzelnen Tasks Kommentare, Workarounds und Testbeschreibungen verfassen und zusätzlich Dateipfade für weitere Materialien, wie z.B. Screenshots von Fehlermeldungen, hinterlegen. Einen hohen Stellenwert nimmt dabei die Spalte der Kommentare ein, denn häufig erfolgen Rückfragen durch die Entwickler, da die Rückmeldungen der Anwender keiner einheitlichen Form folgen und teilweise notwendige Informationen zur Problemlösung fehlen. Zu beachten ist, dass auch zur aktuellen produktiv laufenden Version, Anfragen von Anwendern eintreffen, die nach dem identischen Schema verarbeitet werden. Diese Vorgehensweise, der Protokollierung in einer Excel-Datei, Kommunikation über ein Exchange-Postfach und die nicht einheitliche Form der Meldung bindet unnötig Ressourcen und verlängert die Projektzeit teilweise erheblich.	
\end{comment}


\subsection{Wirtschaftlichkeitsanalyse}
\label{wirtschaftlichkeitsanalyse}

\subsection{"Make or Buy"- Entscheidung}
\label{makeOrBuy}

\subsection{Projektkosten}
\label{projektkosten}

\subsection{Amortisationsdauer}
\label{amortisationsdauer}

\subsection{Anwendungsfälle}
\label{anwednungsfaelle}

\subsection{Anforderungen}
\label{anforderungen}



\newpage
\section{Entwurfsphase}
\label{entwurfsphase}
%Der \ac{OTR} ist schon lange in Entwurf und wurde 

\begin{comment}
	Die Kommunikation zwischen Front- und Backend wird durch REST-API Aufrufe realisiert. Um die Daten synchron zu halten wird jede Änderung im Frontend in den \gls{localstorage} des Browsers geschrieben und zeitgleich per HTTP-JSON-PATCH Aufruf ans Backend weitergesendet. Das Backend speichert die neuen oder geänderten Daten im Cache und führt wenn möglich eine Berechnung sowie Validierung durch. Die berechneten Beiträge und Validierungsergebnisse werden anschließend zurückgesendet. Alle Schnittstellen sowie \ac{DTO} werden beim Bauen des Backends automatisch für das Frontend generiert.\\
	Als Versionsverwaltung für dieses Projekt wird \gls{git} mit \gls{gitflow} genutzt. Das \ac{CICD} wird mittels \gls{gitlab} Pipelines realisiert, also eine Möglichkeit automatisch nach jeder Änderung das Projekt zu bauen, zu testen und zu veröffentlichen. Veröffentlicht wird der \ac{OTR} in einem \gls{k8s}-Cluster. Somit muss vorher ein Dockerimage gebaut werden um es dann in einem Container laufen zu lassen. Für jeden Feature-, Develop- und Masterbranch im Git wird jeweils eine eigene Instanz im \gls{k8s} Cluster erstellt, so dass man jeden Stand auf dem Testsystem testen kann.
	Für die Zustandsverwaltung der Daten im Frontend wird der \gls{pinia} benutzt. Dieser ist typsicher, was das Arbeiten mit Typescript erleichtert. Allgemein können mit so einem Store die Daten komponentenübergreifend einheitlich gespeichert werden. In dem \ac{OTR} werden dort wichtige Daten, welche das Backend bei der Initialisierung liefert, gespeichert. Dazu gehören unter anderem Konfigurationen, wie die Oberfläche des \ac{OTR}s aufgebaut werden soll, welche Eingabemöglichkeiten gegeben sind und eventuelle Sitzungsdaten, wenn eine bestehende Sitzung geladen wurde. Außerdem werden berechnete Beiträge sowie eingegebene Daten gespeichert. Alle eingegebenen Daten werden automatisch beim Speichern im Store gleichzeitig ans Backend geschickt. Auf die Daten aus einer anderen Komponente zugreifen kann man mit sogenannten Gettermethoden, welche im Store definiert werden.\\
\end{comment}
\subsection{Zielplattform}
\label{zielplattform}
Das gesamte Projekt soll in einem \gls{Container} innerhalb eines \gls{k8s} Cluster laufen. Demnach ist die Zielplattform eine Linuxdistribution. Durch \gls{k8s} besteht ein Loadbalancing um eine bessere Skalierbarkeit zu ermöglichen. Vorteile dieses Deployments ist außerdem, dass es mehrere Versionen und unterschiedliche Stages gibt die zeitgleich laufen. Dadurch ist die Anwendung besser testbar und neue Versionen können nahtlos veröffentlicht werden.
\subsection{Architekturdesign}
\label{architekturdesign}
Die Backend Architektur besteht aus Microservices nach dem \gls{ddd}-Pattern. Außerdem wird für die Datenpersistenz das \gls{repopattern} benutzt. Im Frontend werden Komponenten benutzt, welche ihre Daten über eine zentrale Datenzustandsverwaltung beziehen. Die Kommunikation zwischen Front- und Backend geschieht über \gls{REST}-Aufrufe. Daten aus dem Frontend werden per JSONPatch Protokoll über die HTTP-Patch-Methode übermittelt. Dadurch können Änderungen auch kleinteilig zwischen Front- und Backend synchronisiert und über das Backend persistiert werden.
\subsection{Entwurf der Benutzeroberfläche}
\label{benutzeroberfläche}
Der designseitige Entwurf der Benutzeroberfläche wurde durch die Marketing-Abteilung per Mockups übernommen. Diese mussten allerdings nicht eins zu eins umgesetzt werden, sondern konnten in Kombination mit den Corporate-Design-Vorlagen als Richtlinie genutzt werden. Außerdem gab es wöchentliche Absprache sowie Absprachen mit einem externen Dienstleister in denen speziell das Design abgestimmt wurde.
\subsection{Datenmodell}
\label{datenmodell}

\subsection{Geschäftslogik}
\label{geschaeftslogik}
Die Logik der Benutzeroberfläche und die technische Umsetzung wurde innerhalb des Entwicklerteams durch Prototyping erarbeitet.\\
Dadurch musste für die Fußzeilenkomponente speziell keine eigene Logik oder Design entworfen werden, es wurde anhand des Entwurfs für das Gesamtprojekt bearbeitet.
\subsection{Maßnahmen zur Qualitätssicherung}
\label{qualitaetssicherung}
Durch die Arbeit mit Gitflow wurde jede Unteraufgabe in einem eigenem Gitbranch bearbeitet. Diese Branches werden erst nach Codereviews vom Entwicklerteam in den develop-Branch überführt. Bei jedem Push werden per Gitlab Pipeline Unittests im Front- und Backend ausgeführt. Im Backend habe ich die bestehenden Unittests erweitert um neu implementierte Logik zu testen. Zu dem wurde im Frontend ein \gls{Linter} konfiguriert um einheitliche Codeguidelines und Codestyling durchzusetzen.\\
Wenn eine Änderung auf den Developbranch gepusht wurde wird diese von einem anderen Entwickler getestet. Außerdem gibt es weitere Mitarbeiter außerhalb des Entwicklerteams die regelmäßig auf dem Testsystem Blackbox-Akzeptanztests durchführen.
\newpage
\section{Implementierungsphase}
\label{implementierungsphase}
\begin{comment}
	In Implementierung verlagern
	
	Beim Speichern wird über mehrere anliegende Systeme ein Angebot in einer Datenbank gespeichert. Wichtig ist dabei die Session, welche von einem der Systeme generiert wird, realisiert als \ac{GUID}. Mithilfe dieser wird ein Angebot im XML-Format in einer Datenbank gespeichert und ist dadurch eindeutig identifizierbar.\\
	
	Dies beinhaltet die gleiche Logik wie die Angebotsspeicherung, bis auf dass eine neue \ac{GUID} erzeugt wird, wodurch dementsprechend ein neuer Eintrag in der Datenbank und im Maklerportal hinterlegt wird. Außerdem muss mit der \ac{GUID}, bzw. Session der Kopie weitergearbeitet werden.\\
\end{comment} 

\subsection{Implementierung der Datenstruktur}
\label{datenstruktur}

\subsection{Implementierung der Benutzeroberfläche}
\label{oberflacheimplemetieren}
Die Benutzeroberfläche wird mit \gls{vue}, \gls{vuetify} und \gls{typescript} realisiert. Außerdem gibt es eine interne Komponentenbibliothek als \gls{npm}-Paket, welche Vuetify-Elemente kapselt und im Corporate Design und mit geringfügiger aber erweiterbarer Funktionalität anbietet. \gls{vue} großer Vorteil ist es, dass es die Oberfläche reaktiv macht. Das bedeutet, dass wenn sich Daten die mit einem Oberflächenelement verbunden sind ändern, wird dieses Element automatisch aktualisiert. Dies geschieht über das Observer-Designpattern. \gls{vuetify}, bzw. die Komponentenbibliothek hilft bei einem einheitlichem Design und erleichtert die Einbindung. Die durch \gls{typescript} bereit gestellte Typsicherheit macht das Arbeiten im Team, das Benutzen von Bibliotheken, vorausgesetzt diese haben Typescriptdatentypen deklariert sowie die Wartung und zukünfitge Erweiterungen einfacher.\\
Für die Zustandsverwaltung der Daten im Frontend wird der \gls{pinia} benutzt, eine Erweiterung des \gls{vue}-Stores. Dieser ist typsicher, was das Arbeiten mit Typescript erleichtert. Allgemein können mit so einem Store die Daten Komponenten übergreifend einheitlich gespeichert werden. In dem \ac{OTR} werden dort wichtige Daten welche das Backend bei der Initialisierung liefert gespeichert. Dazu gehören unter anderem Konfigurationen wie der \ac{OTR} aufgebaut werden soll welche Eingabemöglichkeiten gegeben sind und eventuelle Sitzungsdaten, wenn eine bestehende Sitzung geladen wurde. Außerdem werden berechnete Beiträge sowie eingegebene Daten gespeichert. Alle eingegebenen Daten werden automatisch beim Speichern im Store gleichzeitig ans Backend geschickt. Es gibt mehrere Teilstores, einen allgemeingültigen Store, in welchen Daten gespeichert werden welche tarifunabhängig sind und ein Store für den jeweilig aktiven Tarif. Auf die Daten aus einer anderen Komponente zugreifen kann man mit sogenannten Gettermethoden, welche im Store definiert werden.\\
Eine weitere Funktion des Stores ist ein Benachrichtigungssystem. Hierbei wird bei jeder Änderung von bestimmten Daten ein Ereignis losgetreten, so dass dann eine Benachrichtigung, z.B. ein Fehler oder eine Erfolgsnachricht bei einer Aktion, ausgegeben wird.\\

Als Fußzeilencontainer war die \textit{<footer>} Komponente von \gls{vuetify} gegeben. Die einzelnen Schaltflächen habe ich mit \textit{<km-button>} umgesetzt. Dieser ist in der besagten \ac{KM} Komponentenbibliothek gegeben. Eine Besonderheit ist hierbei der Button für den Dokumentenversand. Da der Dokumentenversand relativ viel Eigenlogik benötigt und schon in einer vorherigen \ac{OTR} Generation vorhanden war, gibt es diesen als eigenes \gls{npm}-Paket. Im Endeffekt wird in dieser Komponente allerdings auch nur der \textit{<km-button>} verwendet.\\

Die Angebot- bzw. Kopiespeichern-Schaltfläche soll nur vorhanden sein, wenn der \ac{OTR} über das Maklerportal aufgerufen wurde. Für so etwas bietet \gls{vue} ein konditionales Rendering als \gls{HTML}-Attribut an. Im \gls{pinia} wird überprüft, ob das Angebot speicherbar sein soll und dieser Wert wird an das Attribut übergeben. Beim Drücken auf den Knopf wird ein \gls{fetch} Aufruf ans Backend an den Management-Controller an die SaveOrder bzw SaveOrderCopy Action \marginline{Zu Spezifisch mit den Namen?} gemacht. Dieser ist durch automatisch generierte Klassen anhand der Schnittstellendefinition des Backends gekapselt. Diese kümmern sich um Fehlerbehandlung anhand des \gls{HTTP}-Statuscodes, erwarten die Argumente als richtigen Datentyp und geben die Response als eigenen Datentyp zurück. Ein Angebot wird anhand einer \ac{GUID} gespeichert, welche die aktuelle Sitzung des \ac{OTR} repräsentiert. Außerdem wird die Identifikationsnummer des angemeldeten Maklers, bzw. Vertriebspartners benötigt. Dementsprechend werden diese Daten mit ans Backend übergeben. Dieses kann sich anhand der Session die Angebotsdaten aus dem Cache laden und diese dann an das \ac{TRG} zum Speichern weiterschicken. Auf der Seite des Frontends passiert beim Kopiespeichern das Gleiche wie bei der normalen Speicherung, außer dass ein anderer Endpunkt aufgerufen wird.\\

Die Angebot- und Antragdruck-Schaltflächen sind zwar immer vorhanden, sind allerdings nicht immer aktiviert, bzw. benutzbar. Diese Logik setzt sich aus zwei Bedingungen zusammen. Zum einem muss überhaupt für den aktuellen Tarif der Antrags- bzw. Angebotsdruck aktiviert sein. Zum anderen ist das Drucken erst möglich, wenn ein Beitrag berechnet werden konnte. Diese Bedingungen werden im \gls{pinia} überprüft und in den Schaltfächenkomponenten in dem disabled-Attribut benutzt. Beim Drücken der Buttons wird ein Backend Aufruf an die Drucklogik gestartet. Dieser erwartet die Sessionguid, damit sich das Backend die benötigten Daten aus dem Cache laden kann. Zurückgeliefert wird ein PDF-Dokument als Filestream \marginline{Prüfen wie genau?}. Das wird dann im Frontend aufbereitet, so dass automatisch ein PDF-Dokument heruntergeladen wird. Da die Logik hierbei im Frontend zwischen dem Antrag und Angebot sich nur im aufgerufenem Endpunkt unterscheidet, ist die Logik zusammengefasst und der Aufruf des Backends kann als Action per Parameter reingegeben werden.\\
In die Fußzeile kommt außerdem noch das Impressum und das Copyright. Das Impressum ist hierbei eine Verlinkung auf das Impressum der Homepage von \ac{KM}.
\subsection{Implementierung der Geschäftslogik}
\label{geschaeftslogikimplementieren}
Die Geschäftslogik, bzw. das Backend ist eine \gls{net} \gls{api}. Als Programmiersprache wird dementsprechend C# 10 benutzt. Das Backend persistiert Daten in einem Storage oder Cache. \marginline{writer lock,cache,storage, repository pattern} Da die Kommunikation zwischen dem Backend und den anliegenden Systemen größtenteils über XML läuft, werden die gegebenen XML-Schemata eingelesen und beim Bauen des Backends als C# Klassen generiert. Dadurch können die Daten einheitlich und einfach geparst und ausgelesen werden. Allerdings kann es trotzdem vorkommen, dass Daten in anderen Datentypen verarbeitert werden. Um diese dann in das besagte XML-Format zu stecken \marginline{besser formulieren} wird ein \gls{automapper} genutzt, welchen man nur konfigurieren muss um die Daten in der jeweils andere Klasse speichern zu können.\\

Das Backend ist im \ac{ddd} -Pattern aufgebaut. Dementsprechend gibt es mehrere Services, welche jeweils in mehrere Projekte aufgeteilt sind. Unter anderem gibt es im Backend Schnittstellen für Anträge, Beitragsberechnung, Tarifkonfigurationen und allgemeine Logik unter Management. Jede dieser Schnittstellen haben nach dem \ac{ddd} ein eigenes Projekt für die eigentliche Schnittstelle, für die Infrastruktur und für die Logik. Außerdem wird die Logik für die verschiedenen Versicherungstarife aufgeteilt, da sie sich dort unterscheiden kann. Die unterschiedliche Logik der Tarife wird nach Strategy Pattern dynamisch je nach aktivem Tarif benutzt und per Dependency Injection Pattern reingegeben. Für die Fußzeilenkomponente ist nur der Managementservice interessant, da dort allgemeingültige und tarifunspezifische Logik zu finden ist.\\

Zum Speichern und Kopieren von Angeboten wird die Session als \ac{guid} benötigt. Anhand dieser Sessionguid, wird sich die Konfiguration und die gespeicherten Angebotsdaten aus dem Repositorycache geladen. Sowohl beim Speichern als auch beim Kopieren wird größtenteils die gleiche Logik benutzt. So wird bei beiden Varianten die Sessionspeicherungsschnittstelle des \ac{TRG}s aufgerufen, welches einen Eintrag in einer Datenbank hinterlegt oder falls schon einer mit dieser Sessionguid vorhanden ist, bearbeitet. Anschließend wird das Angebot im Repositorystorage gespeichert. \\ Die Besonderheit bei der Kopie ist, dass vor dem Aufruf ans \ac{TRG} eine neue \ac{guid} erzeugt wird. Dadurch wird ein neuer Eintrag in die Datenbank geschrieben. Außerdem muss das Angebot mit der neuen Sessionguid auch im Cache gespeichert werden. Die neue Sessionguid wird nun ans Frontend zurückgegeben, damit dort mit der Kopie weitergearbeitet wird.\\


\section{Projektabschluss}
\label{Projektabschluss}

\subsection{Abnahme}
\label{Abnahme}
Durch die regelmäßigen Absprachen und Codereviews durch das Entwicklungsteam sowie die wöchentlichen Absprachen bezüglich der Oberfläche gab es keine formale Abnahme der Fußzeilenkomponente. Alle anfallenden Anforderungen konnten im Laufe der Implementierung umgesetzt werden. Ausgenommen davon sind Anforderungen die kurz vor Abschluss des Projekts aufgekommen sind. Auf diese wird in \ref{ausblick} weiter eingegangen. Außerdem wurde regelmäßig die Funktionalität getestet, so dass jeder Bestandteil einzeln getestet und abgenommen wurde. 
\subsection{Soll-Ist-Vergleich}
\label{sollIstVgl}
In dem Soll-Ist-Vergleich wird gezeigt, wie schon in \ref{benutzeroberfläche} \nameref{benutzeroberfläche} erwähnt, dass durch die bestehenden Mockups und Absprachen weniger Zeit bei der Entwicklung und dem Entwerfen der Oberfläche benötigt wurde, als geplant. Diese gewonnene Zeit ist allerdings in die wöchentlichen Absprachen zu der Oberfläche, die Analyse bestehender Codebasen sowie Implementierung der Frontendlogik geflossen.\\

\begin{tabular}{|l|r|r|r|}
	\hline
	\rowcolor{blue!70}\textbf{Projektphasen }                                   & \textbf{Soll in Std} & \textbf{Ist in Std} & Differenz \\ \hline
	\rowcolor{blue!10}Tägliche Absprachen mit Entwicklungsteam \& Auftragsgeber &                    1 &                   1 &         0 \\ \hline
	\rowcolor{blue!10}Wöchentliche Absprache zur Oberfläche                     &                    0 &                   1 &        +1 \\ \hline
	\rowcolor{blue!50}\textbf{Ist-Analyse: }                                    &           \textbf{5} &          \textbf{6} &        +1 \\ \hline
	\rowcolor{blue!10}Analyse vorhandener Codebasen und Funktionalitäten        &                    2 &                   3 &        +1 \\ \hline
	\rowcolor{blue!10}Gespräche mit dem Auftragsgeber                           &                    3 &                   3 &         0 \\ \hline
	\rowcolor{blue!50}\textbf{Sollkonzept:  }                                   &           \textbf{5} &          \textbf{4} &        -1 \\ \hline
	\rowcolor{blue!10}Planung des Soll-Zustands Projekt allgemein               &                    2 &                   2 &         0 \\ \hline
	\rowcolor{blue!10}Planung des Soll-Zustands Fußzeilenfunktionalität         &                    1 &                   1 &         0 \\ \hline
	\rowcolor{blue!10}Abstimmung \& Sichtung Workflows und Oberflächenentwürfe  &                    2 &                   1 &        -1 \\ \hline
	\rowcolor{blue!50}\textbf{Realisierung: }                                   &          \textbf{31} &         \textbf{30} &        -1 \\ \hline
	\rowcolor{blue!30}\textbf{Frontend  }                                       &          \textbf{10} &         \textbf{12} &        +2 \\ \hline
	\rowcolor{blue!10}Entwicklung Oberfläche                                    &                    4 &                   1 &        -3 \\ \hline
	\rowcolor{blue!30}\textbf{Backend}                                          &                   17 &                  17 &         0 \\ \hline
	\rowcolor{blue!50}\textbf{Dokumentation }                                   &                   19 &                  19 &         0 \\ \hline
	\rowcolor{blue!50}\textbf{Testen \& Abnahme }                               &                    8 &                   8 &         0 \\ \hline
	\rowcolor{blue!70}\textbf{Gesamt}                                           &                   69 &                  69 &         0 \\ \hline
\end{tabular}
\captionof{table}{Vergleich gebrauchte Zeit der Projektphasen}\leavevmode 

 Von dem geplanten Umfang des Teilprojekts wurde alles erfüllt. Zudem wurde die in \ref{abweichung} \nameref{abweichung} erwähnte Angebotskopiespeicherung implementiert.
\subsection{Gewonnene Erkenntnisse}
\label{erkenntnisse}
Die Umsetzung dieses Projekts brachte wertvolle Erkenntnisse. Es wurde das erste Mal mit \gls{vue} 3 entwickelt, was im Gegensatz zum Vorgänger einige Änderungen mit sich brachte. Dadurch wurde im Bereich der Frontendentwicklung viel neues gelernt. \gls{vuetify} bzw. die Komponentenbibliothek half bei einem einheitlichem Design und erleichterte die Einbindung durch gestellte Vorlagen. Die durch \gls{typescript} bereit gestellten Typsicherheit machte das Arbeiten im Team einfacher. Das Benutzen von Bibliotheken, vorausgesetzt diese haben Typescriptdatentypen deklariert, wurde auch erleichtert. Außerdem wird sich die Wartung und die Umsetzung zukünftiger Erweiterungen problemloser gestalten.\\ 
Dazu kommt, dass das erste Mal mit gegebenen Oberflächenmockups gearbeitet wurde, wodurch das Designen und Entwerfen der Oberfläche erleichtert wurde. Weiterhin ist durch die Konzeption des Gesamtprojekts von einem Softwarearchitekten eine einheitliche Struktur erkennbar, was das Implementieren neuer Logik und das Anpassen oder Erweitern bestehender Logik einfacher gestaltete.
\subsection{Ausblick}
\label{ausblick}
Als neuste Generation der \ac{OTR} sollen nun alle neuen Tarife anhand dieser Basis implementiert werden. Momentan ist geplant mindestens einen alten Tarif in die neue Basis zu übertragen.\\
Die Konfigurationen für alle Tarife sind zur Zeit in verschiedenen Systemen verstreut. Geplant ist es diese in einer NoSQL Datenbank mit Anbindung zum neuen \ac{OTR} zu speichern, so dass der \ac{OTR} die zentrale Stelle für Tarifkonfigurationen wird.\\
Für die Fußzeilenkomponente sind noch kleinere Erweiterungen ausstehend, die nicht im Rahmen des Abschlussprojektes umgesetzt werden konnten oder nicht geplant waren. Dazu gehört eine Ladeanimation und Eingabesperre im Frontend, solange auf das Drucken eines Dokuments gewartet wird. Hinzufügend sind kleinere Refactorings im Backend sowie ein Refactoring der Datenzustandsverwaltung im Frontend geplant.


% Literaturverzeichnis
\clearpage
\singlespacing
\pagenumbering{Roman}

\renewcommand\refname{Literaturverzeichnis}
\bibliography{literatur/sammlung}

\onehalfspacing
%\fancyhead[L]{Anhang}
\appendix

\section{Anhänge}

\begin{figure}[!htbp]
	\includegraphics[width=\textwidth, height=\textheight, keepaspectratio]{anhang/usecase_footer.png}
	\caption{Anwendungsfalldiagramm der Fußzeilenkomponente}
	\label{usecasefooter}
\end{figure}
\begin{figure}[!htbp]
	\includegraphics[width=\textwidth, height=\textheight, keepaspectratio]{anhang/mockup_footer.png}
	\caption{Mockup-Design der Fußzeile}
	\label{mockup}
\end{figure}
\begin{figure}[!htbp]
	\includegraphics[width=\textwidth, height=\textheight, keepaspectratio]{anhang/actual_footer.png}
	\caption{Tatsächliches Design der Fußzeile}
	\label{actualfooter}
\end{figure}
\newpage
\onehalfspacing

\newcommand{\abkvz}{Abkürzungsverzeichnis}
\section*{\abkvz}
\markboth{\abkvz}{\abkvz}
\begin{acronym}
	
	\acro{KM}[K\&M]{Konzept und Marketing GmbH}
	\acro{SPA}[SPA]{Single-Page-Application}
	\acro{OTR}[OTR]{Onlinetarifrechner}
	\acro{GUID}[GUID]{Globally Unique Identifier}
	\acro{TG}[TG]{Tarifrechner-Gateway}
	\acro{TRG}[TRG]{Tarifrechner-REST-Gateway}
	\acro{CICD}[CI/CD]{Continuous Integration/Continuous Delivery}
\end{acronym}
\newpage
\printglossaries
\end{document}