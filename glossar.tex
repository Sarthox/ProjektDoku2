\newglossaryentry{k8s}{name={Kubernetes}, description={(Orchestrator für Containeranwendungen)}}
\newglossaryentry{vue}{name={Vue.js}, description={Ein clientseitiges Javascript-Framework nach dem Model-View-ViewModel-Entwurfsmuster für das Erstellen von Webanwendungen}}
\newglossaryentry{vuetify}{name={Vuetify}, description={Eine Vue.js Oberflächenkomponentenbibliothek}}
\newglossaryentry{typescript}{name={TypeScript}, description={Eine typsichere Erweiterung von JavaScript}}
\newglossaryentry{npm}{name={npm}, description={Ein Paketmanager für Node.js Pakete}}
\newglossaryentry{pinia}{name={Pinia-Store}, description={Eine Datenzustandsverwaltung für Vue.js welche den Localstorage des Browsers verwendet}}
\newglossaryentry{HTTP}{name={HTTP}, description={Hypertext Transfer Protocol; Protokoll für Datenübertragung}}
\newglossaryentry{fetch}{name={Javascript-fetch}, description={Asynchroner Aufruf an ein Backend}}
\newglossaryentry{HTML}{name={HTML}, description={Hypertext Markup Language; textbasierte Auszeichnungssprache um Inhalte in Webbrowsern darzustellen}}
\newglossaryentry{Container}{name={Container}, description={Laufende Instanz eines Softwarepakets im Dockerumfeld}}
\newglossaryentry{helm}{name={Helm}, description={Packetmanager und Hilfstool für Kubernetes}}
\newglossaryentry{repo}{name={Repository}, description={Verzeichnis zur versionierten Speicherung von Quellcode}}
\newglossaryentry{repopattern}{name={Repository-Pattern}, description={Ein Designpattern, welches die Logik der Datenspeicherung- und beschaffung kapselt, so dass eine zentrale Stelle für Datenquellen besteht}}
\newglossaryentry{ddd}{name={Domain-Driven Design}, description={Modellierung der Software nach umzusetzenden Fachlichkeiten der Anwendungsdomäne}}
\newglossaryentry{REST}{name={REST}, description={Reprensentational State Transfer; Basierend auf HTTP; ein Paradigma zur Kommunikation zwischen Systemen}}
\newglossaryentry{Linter}{name={Linter}, description={Ein Tool zur statischen Codeanalyse}}
\newglossaryentry{automapper}{name={Automapper}, description={Eine Bibliothek mit der eine Objekt-zu-Objekt Datenübertragungsstrategie konfiguriert werden kann}}
\newglossaryentry{net}{name={.NET 6.0}, description={Eine Softwareplattform für das Entwickeln und Ausführen von Anwendungen}}
\newglossaryentry{api}{name={API}, description={Application Programming Interface, Programmierschnittstelle; Ein Programmteil der zur Anbindung für andere Systeme zur Verfügung gestellt wird}}
\newglossaryentry{di}{name={Dependency Injection}, description={Ein Entwurfsmuster welches Abhängigkeiten von Objekten zur Laufzeit auflöst}}
\newglossaryentry{localstorage}{name={Localstorage}, description={Ein persistenter, clientseitiger Datenspeicher im Webbrowser}}
\newglossaryentry{git}{name={Git}, description={Dezentrale Versionverwaltung}}
\newglossaryentry{gitflow}{name={GitFlow}, description={Ein Modell für das Benutzen von Gitbranches, in welchem eigenständige Anforderungen in eigene Branches aufgeteilt werden}}
\newglossaryentry{gitlab}{name={GitLab}, description={Ein Anbieter von Git mit DevOps-Funktionen}}