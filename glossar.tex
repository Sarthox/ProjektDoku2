\newglossaryentry{k8s}{name={Kubernetes}, description={Orchestrator für Containeranwendungen}, text={\underline{Kubernetes}}}
\newglossaryentry{vue}{name={Vue.js}, description={Ein clientseitiges Javascript-Framework nach dem Model-View-ViewModel-Entwurfsmuster für das Erstellen von Webanwendungen}, text={\underline{Vue.js}}}
\newglossaryentry{vuetify}{name={Vuetify}, description={Eine Vue.js Oberflächenkomponentenbibliothek}, text={\underline{Vuetify}}}
\newglossaryentry{typescript}{name={TypeScript}, description={Eine typsichere Erweiterung von JavaScript}, text={\underline{TypeScript}}}
\newglossaryentry{npm}{name={npm}, description={Ein Paketmanager für Node.js Pakete}, text={\underline{npm}}}
\newglossaryentry{pinia}{name={Pinia-Store}, description={Eine Datenzustandsverwaltung für Vue.js welche den Localstorage des Browsers verwendet}, text={\underline{Pinia-Store}}}
\newglossaryentry{HTTP}{name={HTTP}, description={Hypertext Transfer Protocol; Protokoll für Datenübertragung}, text={\underline{HTTP}}}
\newglossaryentry{fetch}{name={Javascript-fetch}, description={Asynchroner Aufruf an ein Backend}, text={\underline{JavaScript-Fetch}}}
\newglossaryentry{HTML}{name={HTML}, description={Hypertext Markup Language; textbasierte Auszeichnungssprache um Inhalte in Webbrowsern darzustellen}, text={\underline{HTML}}}
\newglossaryentry{Container}{name={Container}, description={Laufende Instanz eines Softwarepakets im Dockerumfeld}, text={\underline{Container}}}
%\newglossaryentry{helm}{name={Helm}, description={Packetmanager und Hilfstool für Kubernetes}, text={\underline{Helm}}}
\newglossaryentry{repo}{name={Repository}, description={Verzeichnis zur versionierten Speicherung von Quellcode}, text={\underline{Repository}}}
\newglossaryentry{repopattern}{name={Repository-Pattern}, description={Ein Designpattern, welches die Logik der Datenspeicherung und -beschaffung kapselt, so dass eine zentrale Stelle für Datenquellen besteht}, text={\underline{Repository-Pattern}}}
\newglossaryentry{ddd}{name={Domain-Driven Design}, description={Modellierung der Software nach umzusetzenden Fachlichkeiten der Anwendungsdomäne}, text={\underline{Domain-Driven Design}}}
\newglossaryentry{REST}{name={REST}, description={Reprensentational State Transfer; Basierend auf HTTP; ein Paradigma zur Kommunikation zwischen Systemen}, text={\underline{REST}}}
\newglossaryentry{Linter}{name={Linter}, description={Ein Tool zur statischen Codeanalyse}, text={\underline{Linter}}}
%\newglossaryentry{automapper}{name={Automapper}, description={Eine Bibliothek mit der eine Objekt-zu-Objekt Datenübertragungsstrategie konfiguriert werden kann} , text={\underline{Automapper}}}
\newglossaryentry{net}{name={.NET}, description={Eine Softwareplattform für das Entwickeln und Ausführen von Anwendungen}, text={\underline{.NET}}}
\newglossaryentry{api}{name={API}, description={Application Programming Interface, Programmierschnittstelle; ein Programmteil der zur Anbindung für andere Systeme zur Verfügung gestellt wird}, text={\underline{API}}}
\newglossaryentry{di}{name={Dependency Injection}, description={Ein Entwurfsmuster welches Abhängigkeiten von Objekten zur Laufzeit auflöst}, text={\underline{Dependency Injection}}}
\newglossaryentry{localstorage}{name={Localstorage}, description={Ein persistenter clientseitiger Datenspeicher im Webbrowser}, text={\underline{Localstorage}}}
\newglossaryentry{git}{name={Git}, description={Dezentrale Versionverwaltung}, text={\underline{Git}}}
\newglossaryentry{gitflow}{name={GitFlow}, description={Ein Modell für das Benutzen von Gitbranches, in welchem eigenständige Anforderungen in eigene Branches aufgeteilt werden}, text={\underline{GitFlow}}}
\newglossaryentry{gitlab}{name={GitLab}, description={Ein Anbieter von Git mit DevOps-Funktionen}, text={\underline{GitLab}}}
\newglossaryentry{NSwag}{name={NSwag}, description={Ein Tool um Swagger OpenApi Beschreibungen zu generieren}, text={\underline{NSwag}}}
\newglossaryentry{distributedcache}{name={Distributed Cache}, description={Verteilter Cache, Speicher der sich über mehrere Instanzen spannt}, text={\underline{Distributed Cache}}}
\newglossaryentry{php}{name={PHP}, description={Eine Skriptsprache die zur Erstellung dynamischer Webseiten dient}, text={\underline{PHP}}}